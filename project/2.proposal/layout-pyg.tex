
%% One page layout:
% Margins: left 40mm, rigt 25mm, top and bottom 25mm
% (latex adds extra 1in)
\documentclass[12pt,a4paper]{report}
\setlength\textwidth{145mm}
\setlength\textheight{247mm}
\setlength\oddsidemargin{15mm}
\setlength\evensidemargin{15mm}
\setlength\topmargin{0mm}
\setlength\headsep{0mm}
\setlength\headheight{0mm}
% \openright zar�d�, aby n�sleduj�c� text zac�nal na prav� strane knihy
\let\openright=\clearpage

%% Two pages layout:
% \documentclass[12pt,a4paper,twoside,openright]{report}
% \setlength\textwidth{145mm}
% \setlength\textheight{247mm}
% \setlength\oddsidemargin{15mm}
% \setlength\evensidemargin{0mm}
% \setlength\topmargin{0mm}
% \setlength\headsep{0mm}
% \setlength\headheight{0mm}
% \let\openright=\cleardoublepage

\usepackage[utf8]{inputenc}
\usepackage{graphicx}
\usepackage{amsthm}

% PYGMENTIZE
%pygmentize_options: -O style=vs

\usepackage[unicode]{hyperref}   % Mus� b�t za vsemi ostatn�mi bal�cky
\hypersetup{pdftitle=Implementing control flow resolution in dynamic language}
\hypersetup{pdfauthor=Step�n Sindel�r}

%%% Small styling hacks

% Tato makra presvedcuj� m�rne oskliv�m trikem LaTeX, aby hlavicky kapitol
% s�zel pr�cetneji a nevynech�val nad nimi spoustu m�sta. Smele ignorujte.
\makeatletter
\def\@makechapterhead#1{
  {\parindent \z@ \raggedright \normalfont
   \Huge\bfseries \thechapter. #1
   \par\nobreak
   \vskip 20\p@
}}
\def\@makeschapterhead#1{
  {\parindent \z@ \raggedright \normalfont
   \Huge\bfseries #1
   \par\nobreak
   \vskip 20\p@
}}
\makeatother

% Toto makro definuje kapitolu, kter� nen� oc�slovan�, ale je uvedena v obsahu.
\def\chapwithtoc#1{
\chapter*{#1}
\addcontentsline{toc}{chapter}{#1}
}




\usepackage[usenames,dvipsnames,svgnames,table]{xcolor}
\usepackage{mdframed}

\newcommand{\question}[1] {
    \begin{mdframed}
        \emph{\large{#1}}
    \end{mdframed}
}

\begin{document}

% Trochu volnejs� nastaven� delen� slov, nez je default.
\lefthyphenmin=2
\righthyphenmin=2

%%% Tituln� strana pr�ce

\pagestyle{empty}
\begin{center}

\large

Macquarie University

\medskip

Department of Computing

\vfill

{\bf\Large PROJECT PROPOSAL}

\vfill

% N�zev pr�ce presne podle zad�n�
{\LARGE\bfseries Static Type Analysis of Dynamically Typed
Programming Language}

\vfill

\begin{tabular}{rl}
Author: & Stepan Sindelar \\
\noalign{\vspace{2mm}}
Student ID: & 43600220 \\
\noalign{\vspace{2mm}}
Supervisor of the project: &  Matthew Roberts \\
\end{tabular}

\vfill

\end{center}

\newpage

\include{content}

%%% Seznam pouzit� literatury
\def\bibname{Bibliography}
\addcontentsline{toc}{chapter}{\bibname}

  \bibliographystyle{ieeetr}
  \bibliography{../../thesis/bibliography}

\openright
\end{document}
