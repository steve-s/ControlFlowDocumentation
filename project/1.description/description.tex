\documentclass[a4paper,twoside]{article}
\usepackage[utf8]{inputenc}
\usepackage{a4wide}

% umoznuje nastavit margins
\usepackage[hmargin=1.5cm,vmargin=2cm]{geometry}

% for multicolumn environmnet
\usepackage{multicol}

% compressed versions of listings, e.g. begin{itemize*} instead of \begin{itemize}
\usepackage{mdwlist}

\usepackage{url}

% this changes labeling of second level lists
\renewcommand{\labelenumii}{\arabic{enumii} $\models$ }
% other counters: \alph, \Alph, \roman

\usepackage{graphicx}

\providecommand{\OO}[1]{\operatorname{O}\bigl(#1\bigr)}

\setlength{\parindent}{0cm} % Default is 15pt.


\author{Štěpán~Šindelář}
\title{Project Description: Static Type Analysis of Dynamically Typed Programming Language}

\begin{document}

  \maketitle

  \subsection*{Introduction}  
  There are several studies showing poor success rate of software projects, for example \cite{ellis2008impact}. 
  Not only are software projects delivered late or they fail completely before 
  they can be even released, but often released software contains errors and 
  security vulnerabilities. One way we can improve this is providing developers 
  with tools that are capable of automatic analysis of the software they 
  are developing for potential errors.
  
  \subsection*{Project Aim}  
  % Most of the software is developed using programming languages.
  Three out of top ten programming languages in TIOBE index (\cite{tiobe}) fall into 
  a category called dynamic languages. 
  When using statically typed languages, as opposed to dynamic languages, 
  the programmer must choose the data type for each variable and then stick to that type. 
  So it is not possible to assign a floating point number into a variable that was chosen to be of integral numeric type.
  Dynamic programming languages allow dynamic type of variables, meaning that the type of the variable 
  can change depending upon the type of the value we assign to it. 
  This makes dynamic languages easier to use, 
  but at the same time can lead to error prone code, which then leads to 
  problems in later phases of the development and maintenance. 
  The aim of the project is to provide automatic analysis of variables types and 
  thus discover possible type related errors in the source code.
  
  \subsection*{Research Problem}
  Because of their dynamic nature, dynamic languages are more difficult to analyse 
  than statically typed languages, especially if we want the analysis to be 
  reasonably fast so that it can be used in everyday development.   
  There is an ongoing research of the static analysis methods for many different 
  families of programming languages, including dynamic languages. The problem 
  this project is trying to address is to adapt and apply those methods on 
  a real world and widely used programming language PHP. Especially to automatically 
  infer the type information from the source code in order to find possible 
  type mismatch errors and inconsistencies with the explicit type 
  documentation in the comments embedded in the source code.
  
  % Significance and benefits
  \subparagraph*{}
  This kind of analysis can provide a feedback to the developers and can help 
  them to improve their source code quality as well as the correspondence between 
  the source code and documentation comments.
  
  \subsection*{Methodology}
  \begin{itemize}
    \item{} Research into possible methods of static analysis of source code.
    \item{} Analysis of applicability of those methods to the PHP programming language.
    \item{} Adaptation and implementation of the selected method(s).
    \item{} Evaluation on real world open source projects.
    \item{} Final report including the research and analysis as well as 
        documentation of the implementation and evaluation results.
  \end{itemize}
  
  \subsection*{Outcomes}
  Expected outcome of this project is a console application that can process PHP 
  source code files (several at once), analyse them and print out possible 
  errors it discovered. This application will form a basis 
  for a plugin for Visual Studio IDE.
  

  \bibliographystyle{ieeetr}
  \bibliography{../../thesis/bibliography}

\end{document} 

