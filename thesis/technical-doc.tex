    \section{Tables}
        Tables module responsibility is to maintain a database 
        of global symbols that can be referenced from the code.
        The most generic interface \code{ITypeTables} is used 
        only for querying the database. The structure is shown 
        in class diagram \ref{tablesifaces}.
        
\begin{figure}[h]  
  \centering
    \includegraphics*[width=\textwidth,height=\textheight,keepaspectratio]{img/tables-ifaces.png}  
    \caption{The Interface of Type Tables\label{tablesifaces}}
\end{figure}

        The tables need to know a context from which the 
        query is made, because certain names can refer 
        to different symbols in different namespaces 
        for example. The \code{ContextManager} class maintains 
        the current context and the consumers of the \code{ITypeTables} 
        should invoke \code{EnterContext} and \code{LeaveContext} on its 
        \code{ContextManager} to provide the correct context 
        for their queries.

        Global variables and fields provide two pieces type 
        information: \code{ExpectedType} is checked when a 
        value is assigned to that variable and the \code{Type} 
        is used as the type of an expression that accesses 
        the variable. Typically, the expected type is what 
        the PHPDoc states and the ``type'' can be either what 
        PHPDoc states or type inferred from all the assignments 
        made to that variable.
        
        The fact that results of \code{ITypeTables} queries 
        are only abstract type information not related to a 
        concrete element in code, allows us to support built-in 
        functions and classes and also to handle conditional 
        declarations by merging the type information from 
        all the occurrences found. However, this has not 
        been implemented, but the design is prepared for it, 
        shall it be considered an important feature in the future. 
        At the moment, if two or more declarations are found, 
        the Type Tables behave as if they did not know the 
        element at all, which preserves correctness for the 
        price of loosing precision.
        
        The basic implementation of \code{ITypeTables} is 
        \code{Tables} class, which follows a variant of the 
        composite design pattern. It gets a list of other 
        \code{ITypeTables} implementations as its constructor 
        parameter and queries those one by one until the 
        symbol is found, or returns a value that indicates 
        that the symbol was not found. The actual implementations 
        of \code{ITypeTables} include Code Tables discussed 
        later and can include user defined \code{ITypeTables} 
        implementation that provides information about 
        built-in functions and classes.
    
        \subsection{Code Tables}
        Code Tables provide information related to concrete elements in 
        the analysed code. So in this case not abstract type related 
        information is provided, but a reference to the AST element 
        with the declaration. Another difference to Type Tables is that 
        Code Tables can return more results for one symbol name, 
        which is because of conditional declarations. 
        Other than that, the interface is similar to the one of 
        Type Tables.
        
        \subsection{Dependency Resolver}
        Dependency Resolver serves as an adapter of Code Tables interface 
        to the Type Tables interface. However, in the case of routines, 
        the routine's declaration AST element does not provide all the 
        necessary type information straight away. Especially inferred 
        return type, which is results of Type Analysis, will not be 
        available if the analysis has not been performed yet.
        
        By convention, all the necessary information for analysing a 
        routine and the results of the type analysis are stored in an 
        instance of \code{RoutineContext} class in the additional attributes 
        of the routine's declaration AST element. Since this AST element 
        is returned by Code Tables, Dependency Resolver can check if 
        the element has the annotation and if so return it as the result, 
        and if not, it can start analysis of the routine. However, 
        there can be cyclic references between routines, so Dependency 
        Resolver also maintains a list of all the routines that are 
        currently being analysed and if a routine is already in the list, 
        default type information is returned as the query result instead 
        of performing the analysis.
        
        \subsection{RoutineContext}
        
        \code{RoutineContext} class provides information about a routine 
        needed for most of the analysis in order to analyse the routine. 
        All the local variables, parameters and referenced global and 
        static variables are numbered and accessed through their number 
        instead of their name. The data \code{RoutineContext} provides are:
        
        \begin{itemize*}
            \item Lists of
                \begin{itemize*}
                    \item all referenced types in the routine's body,
                    \item all referenced variables (local, global, static).
                \end{itemize*}
            \item Results of the points-to analysis: lists of all variables that
                \begin{itemize*}
                    \item given variable can point to, 
                    \item can be pointed to by another variable, 
                    \item are captured by reference by some lambda function.
                \end{itemize*}
            \item Control Flow Graph.
            \item Signature information -- implementation of \code{IRoutineTypesInfo} from Type Tables.
            \item Results of Type Analysis:
                \begin{itemize*}
                    \item inferred return type.
                \end{itemize*}
        \end{itemize*}