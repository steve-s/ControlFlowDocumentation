
%% One page layout:
% Margins: left 40mm, rigt 25mm, top and bottom 25mm
% (latex adds extra 1in)
\documentclass[12pt,a4paper]{report}
\setlength\textwidth{145mm}
\setlength\textheight{247mm}
\setlength\oddsidemargin{15mm}
\setlength\evensidemargin{15mm}
\setlength\topmargin{0mm}
\setlength\headsep{0mm}
\setlength\headheight{0mm}
% \openright zařídí, aby následující text začínal na pravé straně knihy
\let\openright=\clearpage

%% Two pages layout:
% \documentclass[12pt,a4paper,twoside,openright]{report}
% \setlength\textwidth{145mm}
% \setlength\textheight{247mm}
% \setlength\oddsidemargin{15mm}
% \setlength\evensidemargin{0mm}
% \setlength\topmargin{0mm}
% \setlength\headsep{0mm}
% \setlength\headheight{0mm}
% \let\openright=\cleardoublepage

\usepackage[utf8]{inputenc}
\usepackage{graphicx}
\usepackage{amsthm}

\usepackage[unicode]{hyperref}   % Musí být za všemi ostatními balíčky
\hypersetup{pdftitle=Implementing control flow resolution in dynamic language}
\hypersetup{pdfauthor=Štěpán Šindelář}

%%% Small styling hacks

% Tato makra přesvědčují mírně ošklivým trikem LaTeX, aby hlavičky kapitol
% sázel příčetněji a nevynechával nad nimi spoustu místa. Směle ignorujte.
\makeatletter
\def\@makechapterhead#1{
  {\parindent \z@ \raggedright \normalfont
   \Huge\bfseries \thechapter. #1
   \par\nobreak
   \vskip 20\p@
}}
\def\@makeschapterhead#1{
  {\parindent \z@ \raggedright \normalfont
   \Huge\bfseries #1
   \par\nobreak
   \vskip 20\p@
}}
\makeatother

% Toto makro definuje kapitolu, která není očíslovaná, ale je uvedena v obsahu.
\def\chapwithtoc#1{
\chapter*{#1}
\addcontentsline{toc}{chapter}{#1}
}




\usepackage[usenames,dvipsnames,svgnames,table]{xcolor}
\usepackage{mdframed}

\newcommand{\question}[1] {
    \begin{mdframed}
        \emph{\large{#1}}
    \end{mdframed}
}

\begin{document}
\newcommand{\wthesis}[0]{report}
\newcommand{\wThesis}[0]{Report}

\newcommand{\wchapter}[0]{section}
\newcommand{\wChapter}[0]{Section}

\newcommand{\wsection}[0]{paragraph}
\newcommand{\wSection}[0]{Paragraph}

% Trochu volnější nastavení dělení slov, než je default.
\lefthyphenmin=2
\righthyphenmin=2

%%% Titulní strana práce

\pagestyle{empty}
\begin{center}

\vfill

\centerline{\mbox{\includegraphics{img/mq-logo.png}}}

\vspace{2cm}

{\bf\Large PROJECT REPORT}

\vfill

% Název práce přesně podle zadání
{\LARGE\bfseries Static Type Analysis of a Dynamically Typed
Programming Language}

\vfill

\begin{tabular}{rl}
Author: & Stepan Sindelar \\
\noalign{\vspace{2mm}}
Student ID: & 43600220 \\
\noalign{\vspace{2mm}}
Supervisor of the project: &  Matthew Roberts \\
\noalign{\vspace{2mm}}
Study programme: & exchange student \\
\end{tabular}

\vfill

\end{center}

\newpage
\noindent
I would like to express sincere gratitude to Matthew Roberts for being my 
supervisor and for his patience throughout my work on this project. 
Moreover, I am grateful that I could work on my project at 
Macquarie University and would like to thank to Anthony Sloane, 
Scott Buckley and Shaun Muscat from the Programming Languages 
Research Group for the chance to attend the group meetings 
and discuss my project with them.

\newpage
\vspace{3cm}
\begin{center}
ABSTRACT:
\end{center}

Dynamic programming languages allow us to write code without type 
information and types of variables can change during execution. 
Although easier to use and suitable for fast prototyping, 
dynamic typing can lead to error prone code and is challenging 
for the compilers or interpreters.
Programmers often use documentation comments to provide 
the type information, but the \mbox{correspondence} of the 
documentation and the actual code is usually not checked 
by the tools. In this \wthesis, we focus on one of the most popular dynamic 
programming languages: PHP. We have developed a framework 
for static analysis of PHP code as a part of the Phalanger project -- 
the PHP to .NET compiler. The framework supports any kind of analysis, 
but in particular, we implemented type inference analysis with emphasis 
on discovery of possible type related errors and mismatches between 
documentation and the actual code. The implementation was evaluated 
on real PHP applications and discovered several real errors and 
documentation mismatches with a good ratio of false positives.

\newpage

%%% Strana s automaticky generovaným obsahem diplomové práce. U matematických
%%% prací je přípustné, aby seznam tabulek a zkratek, existují-li, byl umístěn
%%% na začátku práce, místo na jejím konci.

\openright
\pagestyle{plain}
\setcounter{page}{1}
\tableofcontents

%%% Jednotlivé kapitoly práce jsou pro přehlednost uloženy v samostatných souborech
\chapter{Introduction}

    \section{Problem Description}
    Static type analysis of a dynamic language, benefits: 
    compiler optimizations, IDE support - bug hunting analysis.

    \section{Implementation Constraints}
    Use Phalanger front end, provide support for compiler back end 
    and IDE (re-analysis), focus on object oriented PHP 5 style 
    projects.
    
    \section{Thesis structure}
%pygmentize_options: -O startinline=True

\chapter{Analysing PHP Code}

    The PHP programming language first appeared in 1995\cite{phphist}. Over the years 
    the language has evolved and so have the ways how programmers use it. 
    This project focuses on PHP version 5.5\footnote{From this point, 
    if the PHP version is not stated explicitly, it is implicitly 5.5.} 
    and the aim for the analysis 
    is to work well on PHP source code written in an object oriented style, 
    using modern PHP patterns and idioms that are described later in 
    this text. The analysis, however, should provide correct results for 
    any valid PHP code of any PHP version. We do not focus only on websites, but also on 
    PHP libraries and frameworks that by themselves do not contain 
    any PHP files that produce HTML or any other output for the user.
    
    \section{PHP semantics}
    This section describes some important parts of the semantics of 
    the PHP programming language, especially those that represent a 
    challenge for static analysis.
    
    \subparagraph*{}    
    In PHP, local or global variables, object fields and function or 
    method parameters are dynamically typed, which means that they 
    can hold values of completely different types at different 
    times of the execution.
    
    \subsection{Local Variables}
    Local variables in PHP do not need to be declared explicitly. 
    Instead the first usage of a variable is also its declaration. 
    If a variable's value is used before the variable got any 
    value assigned, then the interpreter generates a notice, 
    however the execution continues and value \code{null} is 
    used instead. A variable can get a value assigned to it when it 
    appears on a left hand side of an assignment or when a 
    reference to that variable is created, in which case it gets value 
    \code{null}, but no notice is generated. References are 
    discussed in one of the following subsections.

    The scope of a local variable is always its parent function not the 
    parent code block as in other languages like C or Java. 
    So in the following 
    example, the usage of variable \code{\$y} at the end 
    of the function can generate uninitialized variable notice, 
    however, if \code{\$x} was equal to \code{3}, 
    \code{\$y} will have a value although it 
    was declared in the nested code block.

%pygmentize_begin php
% function foo($x) {
%   if ($x == 3) { $y = 2; }
%   echo $y; // uninitialized variable if x != 3
% }
%pygmentize_end
\begin{Verbatim}[commandchars=\\\{\}]
 \PY{k}{function} \PY{n+nf}{foo}\PY{p}{(}\PY{n+nv}{\PYZdl{}x}\PY{p}{)} \PY{p}{\PYZob{}}
   \PY{k}{if} \PY{p}{(}\PY{n+nv}{\PYZdl{}x} \PY{o}{==} \PY{l+m+mi}{3}\PY{p}{)} \PY{p}{\PYZob{}} \PY{n+nv}{\PYZdl{}y} \PY{o}{=} \PY{l+m+mi}{2}\PY{p}{;} \PY{p}{\PYZcb{}}
   \PY{k}{echo} \PY{n+nv}{\PYZdl{}y}\PY{p}{;} \PY{c+c1}{// uninitialized variable if x != 3}
 \PY{p}{\PYZcb{}}
\end{Verbatim}
    
    \subsection{Global and Local Scope}
    PHP distinguishes two scopes for variables: global scope and 
    local scope. Local scope is a scope of local variables 
    within a user defined routine.         
    Variables that are declared 
    in global scope, that is outside of a user defined routine, 
    are available anywhere in global scope and are called 
    global variables. Global variables are also available 
    in user defined routines as long as they are imported 
    into the routine's scope using the keyword \code{global}.
    
%pygmentize_begin php
% $g1 = 1;  // global variables g1 and g2
% $g2 = 2;
% function foo() {
%   global $g1;
%   echo $g1;   // prints the value of global variable g1
%   $g2 = 4;   // sets the value of local variable g2,
%   // because global variable g2 was not imported, 
%   // its value does not change
% }
%pygmentize_end
\begin{Verbatim}[commandchars=\\\{\}]
 \PY{n+nv}{\PYZdl{}g1} \PY{o}{=} \PY{l+m+mi}{1}\PY{p}{;}  \PY{c+c1}{// global variables g1 and g2}
 \PY{n+nv}{\PYZdl{}g2} \PY{o}{=} \PY{l+m+mi}{2}\PY{p}{;}
 \PY{k}{function} \PY{n+nf}{foo}\PY{p}{()} \PY{p}{\PYZob{}}
   \PY{k}{global} \PY{n+nv}{\PYZdl{}g1}\PY{p}{;}
   \PY{k}{echo} \PY{n+nv}{\PYZdl{}g1}\PY{p}{;}   \PY{c+c1}{// prints the value of global variable g1}
   \PY{n+nv}{\PYZdl{}g2} \PY{o}{=} \PY{l+m+mi}{4}\PY{p}{;}   \PY{c+c1}{// sets the value of local variable g2,}
   \PY{c+c1}{// because global variable g2 was not imported, }
   \PY{c+c1}{// its value does not change}
 \PY{p}{\PYZcb{}}
\end{Verbatim}

    Global variables can be imported and used in any user defined 
    routine. This means that even if we know some type information 
    about a global variable's value at some point in the analysed 
    code (e.g. straight after assignment to that variable), 
    any time another user defined routine is invoked, we 
    have to take into account that the other routine can 
    change the value of the global variable even if we do not 
    pass the global variable to the invoked routine 
    as an argument passed by reference.

    \subsection{Closures}
    PHP also supports anonymous functions. An anonymous function has its 
    own scope as any other function and its local variables are not visible 
    to the scope where it was declared. Variables from the parent 
    scope are available in the closure scope only if they are 
    explicitly imported to its scope and they can be captured 
    by value or by reference. Only the later represents a 
    challenge for the analysis, because any code that can 
    access the closure can invoke it and thus change the 
    values of variables imported to the closure's scope 
    by reference. By invoking a closure, we can influence 
    the values of variables in a completely different 
    and otherwise inaccessible scope.
    
    \subsection{References}
    References in PHP are similar, but not same, as pointers 
    in the C programming language. PHP has a special 
    operator \code{=\&} (assign by reference) that turns 
    the variable on the 
    left hand side into a reference to the 
    variable on right hand side. For example \code{\$a=\&\$b}, 
    after this, any assignment 
    to \code{\$a} will in fact change the 
    value of \code{\$b} and wherever 
    the value of \code{\$a} is used (e.g. in an expression), 
    the value of \code{\$b} is used instead.
    
    The variable where the reference is pointing to is determined 
    in a transitive fashion, which means that if we assign 
    by reference another reference, the new reference will 
    point to where the other reference was pointing to, 
    but the intermediate link is lost. The following example 
    illustrates this.
    
%pygmentize_begin php
% $a =& $b;       // a points to b
% $c =& $a;       // c points to where a points, that is b
% $a =& $d;       // a points to d, but c still points to b
%pygmentize_end    
\begin{Verbatim}[commandchars=\\\{\}]
 \PY{n+nv}{\PYZdl{}a} \PY{o}{=\PYZam{}} \PY{n+nv}{\PYZdl{}b}\PY{p}{;}       \PY{c+c1}{// a points to b}
 \PY{n+nv}{\PYZdl{}c} \PY{o}{=\PYZam{}} \PY{n+nv}{\PYZdl{}a}\PY{p}{;}       \PY{c+c1}{// c points to where a points, that is b}
 \PY{n+nv}{\PYZdl{}a} \PY{o}{=\PYZam{}} \PY{n+nv}{\PYZdl{}d}\PY{p}{;}       \PY{c+c1}{// a points to d, but c still points to b}
\end{Verbatim}

    \subsection{Arrays}
    Arrays in PHP do not have to be homogenous and 
    they can be indexed by either integers or strings.
    In fact, PHP arrays are hash maps rather than arrays 
    in the usual sense and that is also how they are 
    implemented internally. 
    
    String indexed heterogenous arrays are often used 
    as flexible ad-hoc structured data type. 
    Instead of defining a class 
    with required fields, one can use what would be a 
    field name as an index into an array. Such arrays 
    are usually indexed only with finite number of 
    constant string values. 
    
    In this light, it is no 
    surprise that using the subscribe operator 
    \code{[]} with string index on an object instance will 
    access the field with the same name as the index value.

    \subsection{Interesting Control Flow Structures}
    The \code{break} and \code{continue} statements with 
    optional numeric argument are supported in PHP in a 
    similar way as in other standard imperative programming 
    languages. There are, however, important differences 
    to be noted.
    
    Firstly, The numeric argument can be an arbitrary 
    expression in some of the older versions of PHP, in which 
    case we cannot statically determine the target of 
    the jump for the control flow resolution.
    
    Secondly, the \code{switch} statement is considered 
    a loop for the purposes of both \code{break} and 
    \code{continue}. The semantics of \code{break} 
    is intuitive. One of the meaningful use cases is to 
    break a loop from within a \code{switch} by 
    using \code{break 2}. The semantics of 
    \code{continue} statement 
    is perhaps not so intuitive: within a \code{switch} 
    it works the same way as \code{break}.
    
    \subparagraph*{Switch Statement Semantics.} 
    The basic semantics of the \code{switch} statement in PHP is 
    again very similar to that of other standard imperative 
    programming languages. The \code{switch} statement in PHP 
    permits an arbitrary expression as the value to be used 
    for comparison with values of its \code{case} labels. Furthermore, 
    the values of \code{case} labels can also be arbitrary 
    expressions and because we are in a context of dynamic 
    programming language, they can again evaluate to a value 
    of any type.
    
    At runtime, the switch expression is evaluated only once 
    at the beginning, and if it has an undefined value (undefined variable, 
    void function call), then the control flow goes directly 
    to the default item, without evaluating the expressions 
    in the case items. If the value is defined, then it is 
    one by one compared to the values that the 
    \code{case} labels evaluate to. If a \code{case} label evaluates 
    to \code{boolean} value, then it is used to decide whether to 
    jump to that \code{case} item or continue with evaluating 
    the value of the next \code{case} label. Note that the value of 
    switch expression is not compared to the \code{case} label value. 
    If a \code{case} label evaluates to a complex type (\code{object} or \code{array}), 
    it is ignored and evaluation continues with the next \code{case} label. 
    And finally, if a \code{case} label evaluates to an 
    \code{integer}, \code{float} or \code{string} value, it is 
    compared to the switch expression. All these expressions can 
    have side effects due to usage of assignments as expressions 
    or calls of functions with side effects. 
    
    PHP also allows to place the \code{default} label anywhere in between 
    the other \code{case} labels. This can be used for fall-back 
    to or from a \code{case} item as in the following code sample 
    that is abbreviated version of actual code taken from the 
    WordPress\cite{wordpress} code base.
    
%pygmentize_begin php
%switch ( $status ) {
%    default:
%    case 'install':
%        $actions[] = '<a class="install-now" ...';
%        break;
%    case 'update_available':
%        $actions[] = '<a class="install-now" ...';
%        break;
%    case 'newer_installed':
%    case 'latest_installed':
%        $actions[] = '<span class="install-now" ...';
%        break;
%}
%pygmentize_end    
\begin{Verbatim}[commandchars=\\\{\}]
\PY{k}{switch} \PY{p}{(} \PY{n+nv}{\PYZdl{}status} \PY{p}{)} \PY{p}{\PYZob{}}
    \PY{k}{default}\PY{o}{:}
    \PY{k}{case} \PY{l+s+s1}{\PYZsq{}install\PYZsq{}}\PY{o}{:}
        \PY{n+nv}{\PYZdl{}actions}\PY{p}{[]} \PY{o}{=} \PY{l+s+s1}{\PYZsq{}\PYZlt{}a class=\PYZdq{}install\PYZhy{}now\PYZdq{} ...\PYZsq{}}\PY{p}{;}
        \PY{k}{break}\PY{p}{;}
    \PY{k}{case} \PY{l+s+s1}{\PYZsq{}update\PYZus{}available\PYZsq{}}\PY{o}{:}
        \PY{n+nv}{\PYZdl{}actions}\PY{p}{[]} \PY{o}{=} \PY{l+s+s1}{\PYZsq{}\PYZlt{}a class=\PYZdq{}install\PYZhy{}now\PYZdq{} ...\PYZsq{}}\PY{p}{;}
        \PY{k}{break}\PY{p}{;}
    \PY{k}{case} \PY{l+s+s1}{\PYZsq{}newer\PYZus{}installed\PYZsq{}}\PY{o}{:}
    \PY{k}{case} \PY{l+s+s1}{\PYZsq{}latest\PYZus{}installed\PYZsq{}}\PY{o}{:}
        \PY{n+nv}{\PYZdl{}actions}\PY{p}{[]} \PY{o}{=} \PY{l+s+s1}{\PYZsq{}\PYZlt{}span class=\PYZdq{}install\PYZhy{}now\PYZdq{} ...\PYZsq{}}\PY{p}{;}
        \PY{k}{break}\PY{p}{;}
\PY{p}{\PYZcb{}}
\end{Verbatim}
    
    
    \subsection{Conditional Declarations}
    User defined functions, classes, etc. are declared in 
    a global scope in PHP, that is a scope where one can 
    as well place any arbitrary code. Therefore a declaration 
    can be wrapped in any control structure. 
    It is not allowed to redeclare once declared symbol, however.
    
    A typical use case is to dynamically import a file 
    that may provide some functions and then check, 
    using \code{function\_exists}, whether the functions were 
    indeed declared and if not, provide default implementation.
    This is a pre-object-oriented way of doing overriding and 
    is usually not to be found in modern projects. Nonetheless, 
    WordPress still relies on this pattern in parts of its code base.
    
    Although the mentioned pattern could be deemed as 
    reasonable and useful. This feature allows 
    to write very problematic code as in the following example 
    that may or may not crash on fatal errors ``Cannot 
    redeclare foo()'' or ``Call to undefined function foo()'' 
    depending upon the user input.
    
%pygmentize_begin php
%while ($_POST['a'] != 3) {
%   function foo() { return 5; }
%   $_POST['a'] = $_POST['b'];
%}
%echo foo();
%pygmentize_end        
\begin{Verbatim}[commandchars=\\\{\}]
\PY{k}{while} \PY{p}{(}\PY{n+nv}{\PYZdl{}\PYZus{}POST}\PY{p}{[}\PY{l+s+s1}{\PYZsq{}a\PYZsq{}}\PY{p}{]} \PY{o}{!=} \PY{l+m+mi}{3}\PY{p}{)} \PY{p}{\PYZob{}}
   \PY{k}{function} \PY{n+nf}{foo}\PY{p}{()} \PY{p}{\PYZob{}} \PY{k}{return} \PY{l+m+mi}{5}\PY{p}{;} \PY{p}{\PYZcb{}}
   \PY{n+nv}{\PYZdl{}\PYZus{}POST}\PY{p}{[}\PY{l+s+s1}{\PYZsq{}a\PYZsq{}}\PY{p}{]} \PY{o}{=} \PY{n+nv}{\PYZdl{}\PYZus{}POST}\PY{p}{[}\PY{l+s+s1}{\PYZsq{}b\PYZsq{}}\PY{p}{];}
\PY{p}{\PYZcb{}}
\PY{k}{echo} \PY{n+nx}{foo}\PY{p}{();}
\end{Verbatim}
    
                
    \subsection{Auto-loading}
    Historically, in PHP in order to reference any symbol 
    from a different file, one had to import that 
    file explicitly. Newer versions of PHP support  
    customized auto-loading. A used defined routine 
    can be invoked by the runtime every time an 
    undefined class is referenced. 
    The auto-loading routine is then responsible for 
    importing the file(s) that contain the code of the 
    required class. 
    
    Auto-loading routine can use arbitrary logic to 
    determine what file(s) to import, in fact, it can 
    execute arbitrary code. Typical 
    pattern used for example in 
    Zend Framework\cite{zendframework} before namespaces were 
    introduced to PHP is to have a file per class and use 
    class names in form of 
    \code{CodeFolder1\_SubFolderName\_FileName} for 
    a class placed in file 
    \filepath{CodeFolder1\SubFolderName\FileName.php}.
    
    \subsection{PHPDoc Annotations}
    Although not part of the official PHP syntax, 
    there is a widely recognized format for documentation 
    comments of JavaDoc style called PHPDoc. PHPDoc comments 
    may contain type information that cannot be expressed using 
    PHP syntax. For example, PHP allows ``type hints'' 
    for routine parameters, but only for class types, 
    not for primitive types like \code{int}. However, 
    primitive type expectations can be included in the 
    documentation comments. The important difference 
    is that PHP will throw an exception at runtime if 
    a routine is invoked with a parameter of different 
    type than what its ``type hint'' is. The documentation
    comments, on the other hand, are of course ignored 
    by the runtime.
    
    The PHPDoc defines a fairly advanced syntax for expressing 
    type information. It supports multiple 
    primitive and class types, homogenous and heterogenous arrays as well 
    multidimensional arrays, and some constants like \code{false}.
    For example, in the following code the documentation 
    comment tells us that function \code{foo} can return 
    either \code{null}, or \code{false} (but should never 
    return \code{true}), or an array of \code{integer} values.
    
%pygmentize_begin php
% /**
%  * @return null|false|int[]
%  */
% function foo() { ...
%pygmentize_end
\begin{Verbatim}[commandchars=\\\{\}]
 \PY{l+s+sd}{/**}
\PY{l+s+sd}{  * @return null|false|int[]}
\PY{l+s+sd}{  */}
 \PY{k}{function} \PY{n+nf}{foo}\PY{p}{()} \PY{p}{\PYZob{}} \PY{o}{...}
\end{Verbatim}
    
    \section{Static Code Analysis}       
        Static analysis of source code is an analysis that is performed without 
        executing the code. This means that we do not need to have a
        web server, for example, in order to analyse code of a web application. 
        We can also guarantee some properties of the analysis that would not 
        be possible to guarantee if we executed the code. Namely the halting property and 
        upper bounds on time and space complexity. Arbitrary code may not 
        halt if executed, but static analysis of such code can still halt 
        and give results.
        
        Static analysis can be used to get possible types of an expression in 
        a dynamically typed language, to find out expressions that have constant 
        value through constant propagation and many other problems. 
        Static analyses usually do not give accurate solution, but 
        its approximation, which can be an over-approximation or 
        an under-approximation and it is up to the designer and user of the analysis 
        which one is acceptable for his or her\footnote{``His'' or ``he'' 
        should be read as ``his or her'' or ``he or she'' through the rest of the text.} 
        purposes.

        Data Flow Analysis (DFA)\cite{aho1985compilers}, \cite{nielson1999principles} has 
        become a de-facto standard type of analysis for most of the optimizing compilers and 
        other more complex static analyses are either directly based on DFA or on 
        the ideas behind DFA. DFA is also used in Control Flow for Phalanger. 
        In the rest of this section, we provide brief description of 
        firstly DFA in general and then how we have leveraged DFA for 
        the purposes of PHP code analysis.
        
        %---------------------------------------------------------------------------------------
        \subsection{Program State}
        Execution of a program can be seen as a series of transformations of 
        the program state. Each individual program instruction, when executed, 
        can change the program state and produces its \emph{output state}.         
        How exactly is defined by the instruction's semantics and it typically 
        depends on the \emph{input state}, which is a program state produced 
        by the previously executed instruction. 
        
        The goal of a static analysis is to devise some useful piece of information 
        about how instruction $i$ can change the program state, so that it can 
        be used for program optimization or to reveal potentially problematic 
        instructions. For example, if assignment $a=4/b$ always assigns 
        constant value $1$ to $a$, because $b$ happens to be equal to $4$ 
        in any possible program state preceding $a=4/b$, we can change 
        $a=4/b$ to $a=1$, which has the same affect to the program state. 
        If we instead found out that $b$ is always equal to $0$, we would 
        know that this instruction will cause an exception.
        
        The result we expect from a static analysis is, for each instruction 
        in the program, provide some useful property of program state transformation 
        that always holds every time the instruction is executed. 
        The analyses differ in the properties they compute. 
        We will call such computed property a \emph{data-flow}.
        
        
        %---------------------------------------------------------------------------------------
        \subsection{Control Flow Graph.} 
        DFA is typically performed on a control flow graph, 
        although there exist approaches to DFA 
        without explicit control flow graph 
        construction \cite{mohnen2002graph}.
        
        Control flow graph nodes, also called basic blocks, 
        are program statements that are always executed sequentially. 
        Directed edges represent the control flow between basic blocks, 
        for example, jumps in the control flow due to conditionals, 
        \code{goto} statements or any other statements that can change 
        the flow of the program.        
        Control flow graphs usually contain two special nodes: 
        entry node and exit node. The entry node does not have any 
        incoming edges and all the paths lead to the exit node.        
        An example of a control flow graph is given in figure \ref{cfg}.
        
\begin{table}[h]
  \begin{tabular}{ l | m{6cm} }
  \centering
    \includegraphics[scale=0.7]{img/cfg.png}
  &
 
\begin{minipage}{6cm}
%pygmentize_begin php
%   echo 'entry';
%   if ($x == 3)
%       $y = 4;
%   else
%       return 5;
%        
%   while ($y < 7) {
%       echo $y;
%       $y++;
%   }
%
%   return 3;
%pygmentize_end
\begin{Verbatim}[commandchars=\\\{\}]
   \PY{k}{echo} \PY{l+s+s1}{\PYZsq{}entry\PYZsq{}}\PY{p}{;}
   \PY{k}{if} \PY{p}{(}\PY{n+nv}{\PYZdl{}x} \PY{o}{==} \PY{l+m+mi}{3}\PY{p}{)}
       \PY{n+nv}{\PYZdl{}y} \PY{o}{=} \PY{l+m+mi}{4}\PY{p}{;}
   \PY{k}{else}
       \PY{k}{return} \PY{l+m+mi}{5}\PY{p}{;}
        
   \PY{k}{while} \PY{p}{(}\PY{n+nv}{\PYZdl{}y} \PY{o}{\PYZlt{}} \PY{l+m+mi}{7}\PY{p}{)} \PY{p}{\PYZob{}}
       \PY{k}{echo} \PY{n+nv}{\PYZdl{}y}\PY{p}{;}
       \PY{n+nv}{\PYZdl{}y}\PY{o}{++}\PY{p}{;}
   \PY{p}{\PYZcb{}}

   \PY{k}{return} \PY{l+m+mi}{3}\PY{p}{;}
\end{Verbatim}
\end{minipage}

  \\
  \end{tabular}
  \caption{Control flow graph\label{cfg}}  
\end{table}        
        
        %---------------------------------------------------------------------------------------
        \subsection{Transfer functions}
        
        We say that the input state of a statement is associated with 
        the \emph{program point before} the statement and the output state 
        is associated with the \emph{program point after} the statement. 
        Our aim is to calculate \emph{data-flow} value for both 
        program points for each statement, denoted $IN(s)$ and $OUT(s)$ for 
        a statement $s$.
        
        \subsubsection*{Single Statement}
        
        We can express the \emph{data-flow} for \emph{program point after} 
        a statement as a function of the \emph{data-flow} of 
        \emph{program point before} the statement, also called \emph{transfer function}. 
        Formally if $f_t$ is a \emph{transfer function}, then $f_t(IN(s))=OUT(s)$.
        Each type of statement will have a different \emph{transfer function} 
        that will reflect the semantics of the statement. 
        
        \emph{Transfer functions} are often described in the form of inference rules. 
        An example of an inference rule can be 
        ``if the types of expressions $e_1$ and $e_2$ are integers, then the type of 
        expression $e_1+e_2$ is integer''. Those rules can be more formally 
        described with the following notation        
        $$
        \infer{\vdash e_1+e_2 : Integer}{\vdash e_1 : Integer & \vdash e_2 : Integer}
        $$        
        where above the horizontal line we have hypothesis and below is 
        the conclusion. The exact notation is not important, we will be using it 
        intuitively to illustrate the ideas that we describe.
        
        \subsubsection*{Statements Interaction}
        
        The \emph{data-flow} for \emph{program point before} a statement 
        depends on the \emph{data-flow} of \emph{program point after} the 
        last executed statement. Let us consider a basic block $B$ with 
        statements $s_0, s_1, ..., s_n$.
        
        In the case of any $s_i \neq s_0$, that is any statement but 
        the first one, there is only one statement whose execution 
        can precede the execution of $s_i$ and that is the previous statement 
        in the sequence: $s_{i-1}$. So we have $IN(s_i) = OUT(s_{i-1})$ 
        for $i\in{[1..n]}$. 
        
        In fact, thanks to this property, we can define \emph{transfer function} 
        $f_B$ of a basic block as composition of transfer functions of 
        its statements. Formally: 
        
        \[ f_B = f_{s_n} \circ f_{s_{(n-1)}} \circ ... f_{s_0} \]
        
        where $f_{s_i}$ is the transfer function for statement $s_i$.            
        Furthermore, we denote \emph{data-flow values} of basic block as 
        $IN(B)=IN(s_1)$ and $OUT(B)=OUT(s_n)$. From the definitions we can 
        observe simple identities: 
        
        \[ f_B(IN(s_1))=f_B(IN(B))=OUT(B)=OUT(s_n) \]
        
        The first statement $s_1$ of the basic block has to be 
        handled differently. Its execution can be preceded by 
        execution of the last statement of any of 
        the basic blocks $B_1, B_2, ..., B_m$ that precede basic block 
        $B$ in the control flow graph. One of the possibilities to deal with 
        this, is to combine \emph{data-flows} $OUT(B_1), ..., OUT(B_m)$ into 
        single \emph{data-flow} that approximates all of them. 
        Nonetheless, the way the \emph{data-flows} are combined depends 
        on the concrete \emph{data-flow} type and thus 
        should be defined by the analysis. We denote the function to 
        combine \emph{data-flows} as $\mathit{MEET}$. 
        
        \subsubsection*{Equations for Data-Flow}
        
        If we put everything together, we get a set of equations:
        \begin{align*}
            OUT(B_i) &= f_{B_i}(IN(B_i)) \\
            IN(B_i) &= \mathit{MEET}(OUT(P_0), ..., OUT(P_m)) \\ 
        \end{align*}
        where $P_i$ is predecessor of $B_i$ in the control flow graph. 
        The analysis must also define value of $IN(START)$, that is 
        \emph{input state} for the initial node, which does not have 
        any predecessors.
        
        \subsubsection*{Example}
        We will illustrate the system of \emph{data-flow} equations on a 
        simple example. The subject of our analysis will be the 
        type of the variable \code{\$i} in the code sample in figure \ref{dfacfg}.
        
        \emph{Data-flow} will be a set of possible types of variable $i$ and 
        initial \emph{data-flow} of the start node will be an empty set.

\begin{table}[h]
  \begin{tabular}{ l | m{6cm} }
  \centering
    \includegraphics[scale=0.7]{img/dfa-cfg.png}
  &
 
\begin{minipage}{6cm}
%pygmentize_begin php
%   $i = 3;
%   while (foo($i)) {
%       $i = 3.14;
%   }
%   // exit
%pygmentize_end
\begin{Verbatim}[commandchars=\\\{\}]
   \PY{n+nv}{\PYZdl{}i} \PY{o}{=} \PY{l+m+mi}{3}\PY{p}{;}
   \PY{k}{while} \PY{p}{(}\PY{n+nx}{foo}\PY{p}{(}\PY{n+nv}{\PYZdl{}i}\PY{p}{))} \PY{p}{\PYZob{}}
       \PY{n+nv}{\PYZdl{}i} \PY{o}{=} \PY{l+m+mf}{3.14}\PY{p}{;}
   \PY{p}{\PYZcb{}}
   \PY{c+c1}{// exit}
\end{Verbatim}
\end{minipage}

  \\
  \end{tabular}
  \caption{Code for Data-Flow Example\label{dfacfg}}  
\end{table}

        \emph{Transfer functions}: we will assume that function call does 
        not change state, therefore the transfer function of 
        function call is identity -- $OUT(s)=IN(s)$. Assignment of 
        a constant \code{c} of type \code{T} to \code{\$i} changes 
        type of \code{\$i} to \code{T}.
        
        \emph{MEET} operation will be union of the sets of 
        possible types of \code{\$i}.
        
        The set of equations is in this case
        
        \begin{align*}
            OUT(B_1) &= f_{B_1}(IN(B_1))=f_B(\emptyset) \,\,\,\,\,\text{(initial node)} \\
            OUT(B_2) &= f_{B_2}(\mathit{MEET}(OUT(B_1), OUT(B_3))) \\
            OUT(B_3) &= f_{B_3}(OUT(B_2))     
        \end{align*}
        
        Knowing that $f_{B_1}$ and $f_{B_3}$ are constant functions, because 
        the assignment changes type of \code{\$i} to \code{T} without taking 
        the \emph{input data-flow} into account, and $f_{B_2}$ is identity 
        function, we can simplify the equations to

        \begin{align*}
            OUT(B_1) &= \left\{ \mathit{Integer} \right\} \\
            OUT(B_2) &= f_{B_2}(\mathit{MEET}(OUT(B_1), OUT(B_3))) \\
            OUT(B_3) &= \left\{ \mathit{Double} \right\} \\
            \\
            OUT(B_1) &= \left\{ \mathit{Integer} \right\} \\
            OUT(B_2) &= f_{B_2}( \left\{ \mathit{Integer} \right\} \cup \left\{ \mathit{Double} \right\} ) = 
                \left\{ \mathit{Integer}, \mathit{Double} \right\} \\
            OUT(B_3) &= \left\{ \mathit{Double} \right\}
        \end{align*}
        
        With this result we can, for example, check that function \code{foo} 
        is invoked with correct argument type, because from $IN(B_2)$ we 
        know that \code{\$i} at the moment of invocation of \code{foo} 
        can be either of type \code{Integer} or \code{Double}.
                
        \subsection{Finding Solution for the Data-flow Equations}
        
        \subsubsection*{Lattices.}
        The algorithm for finding a solution to a set of data-flow 
        equations is based on algebraic structures called lattices. 
        A lattice is a partially ordered set in which every 
        two elements have a least upper bound, called supremum, 
        and a greatest lower bound, called infimum.
        
        \emph{Bounded lattice} is a lattice that has 
        a greatest element and a least element, 
        usually denoted as $\top$ and $\bot$. 
        A bounded lattice is depicted in figure \ref{lattice}.       
        
\begin{figure}[h]  
  \centering
    \includegraphics{img/lattice.pdf}
  \caption{Bounded lattice with 5 elements.\label{lattice}}    
\end{figure}

        There are several properties of lattices important for DFA.
        
        If we have a finite bounded lattice $(S, \leq_{s})$ and a function 
        $f:S\rightarrow{S}$ that is monotonous, in order words, 
        $\forall{a,b\in{S}}: a\leq_s{b} \Rightarrow f(a)\leq_s{f(b)}$, 
        then $\forall{x\in{S}}$ $\exists{k\in\mathbb{N}}$, such that 
        $f^k(x)=f^{(k+1)}(x)$. $f^k(x)$ is called a fixpoint.
        
        Intuitively, $f$ has to have a fixpoint because 
        for every argument $y$, it must either return 
        $y$ itself, but then $y$ is a fixpoint, or it 
        returns another element that is strictly 
        greater than $y$, but this cannot go on forever, because eventually 
        $f$ will be given $\top$ for which it does not have 
        any other option but returning $\top$ and we 
        have a fixpoint again.
        
        From this property further follows that if we use finite 
        lattice's elements as a domain for transfer functions, 
        and the transfer functions are monotonous, then the set of 
        data-flow equations has a solution, which is a safe 
        approximation of the, in a sense, best solution to the 
        data flow problem. Details can be found in \cite{kildall1973unified}.
        
        Product lattice obtained from two or more lattices 
        is also a lattice, which can simplify the design of 
        data flow analyses.        

        \subsubsection*{The Iterative Algorithm}
        
        The algorithm to find the solution to the data-flow equations 
        initially sets $OUT(B_i)$ for every basic block to the initial 
        \emph{data-flow}, which should be the lowest element of the lattice. 
        Then it iteratively takes a basic block $B_j$ such that 
        its input \emph{data-flow} $IN(B_j)$ has changed or is not 
        initialized yet and computes $OUT(B_j)$, which might change 
        the input \emph{data-flow} of the ancestors of $B_j$. 
        The process is repeated until a the system stabilizes. 
        
        The algorithm can take basic blocks in any order, however, 
        the reverse post order provides the best time complexity 
        \cite{aho1985compilers}.
        
        \subsubsection*{Bit Vectors as Data-flow Representation}

        The performance of the algorithm also depends on the implementation 
        of \emph{data-flow}, the $\textit{MEET}$ operations and the transfer 
        functions. 
        
        \emph{Data-flow} often represents a subset of a set of possible values 
        and the $\textit{MEET}$ operation is either union or intersection.
        For example, subset of all possible types of a variable. 
        Furthermore, we want to calculate the information for all 
        variables not only for one. If we know the number of variables $n$ and 
        types $m$ in advance, we can represent the \emph{data-flow} as 
        a bit-vector where groups of $m$ bits represent a data of 
        single variable and within those bits, value of bit with 
        index $i$ indicates whether the type with index $i$ is in the set. 
        Union or intersection can be implemented using fast bitwise operations.


        % -----------------------------------------------------------------------
        \subsection{Abstraction}
        \note{data-flow value for program point: an abstraction of the set of all possible program 
            states that can be observed for that point. Motivation for abstraction.}
            
        \note{TODO: convert the inference rules to rules with program state 
        and statement on the right side of the line.}
        
        Another example of problem that can be partly solved with static analysis 
        is the sign of integral variables. We can have inference rules of the 
        following form.
        $$
        \infer{\vdash v_1+v_2 : -7 (sign: \ominus)}
        {\vdash v_1 : -10 (sign: \ominus) & \vdash v_2 : 3 (sign: \oplus)}
        $$
        However, the implementation would not be very efficient and we sometimes 
        do not have the full information about variables values, but in some cases 
        we can deduce another less precise, but still useful piece of information 
        by other means. For example, variable of 
        type \code{unsigned integer} will always be positive, we can count on that 
        even if we do not know the actual value. What we 
        can do is to abstract the possible integral values with set 
        $\{0, \ominus, \oplus\}$ with the following meanings         
        \begin{itemize}
            \item $\ominus$ represents all negative integers,
            \item $\oplus$ represents all positive integers,
            \item $0$ represents zero,
        \end{itemize}                
        and rewrite the inference rules as follows:
        $$
        \infer{\vdash v_1+v_2 : \ominus}
        {\vdash v_1 : \ominus & \vdash v_2 : \ominus}
        $$
        Nonetheless, there is another problem. What to do when we have $\ominus$ 
        and $\oplus$ in the hypothesis.
        $$
        \infer{\vdash v_1+v_2 : ?}
        {\vdash v_1 : \ominus & \vdash v_2 : \oplus}
        $$
        The solution is to extend the domain so that it is closed under all operations. 
        We add another element to our set:
        \begin{itemize}
            \item $\top$ represents an unknown value (either positive, negative, or zero).
        \end{itemize}
        Then the rule will be:
        $$
        \infer{\vdash v_1+v_2 : \top}
        {\vdash v_1 : \ominus & \vdash v_2 : \oplus}
        $$
        And for example another rule dealing with $\top$ in hypothesis:
        $$
        \infer{\vdash v_1+v_2 : \top}
        {\vdash v_1 : \ominus & \vdash v_2 : \top}
        $$

        \note{Mention beta and gamma functions formalism for abstractions}
           

        % ----------------------------------------------------------------------
        \subsection{Intraprocedural Analysis}
        So far we have been discussing an analysis of a 
        single function or method\footnote{We will use term 
        routine to designate a global function, static or instance method}. 
        However, if we want to analyse whole program or 
        a library, the interaction between the routines 
        can be taken into account to make it more precise.
        
        \paragraph{Context Sensitive Intraprocedural Analysis.}
        The most precise solution would be to take into account 
        the calling context when analysing a function. 
        In different contexts, the function can be, 
        for example, given different parameters values 
        which may then influence the result of the analysis. 
        More precise result for a specific call site context 
        can be propagated to that call site, 
        yielding another gain in precision when 
        analysing the function that realises the call.
        
        \paragraph{TODO: Region Based Analysis.}
        
        \paragraph*{}
        \note{The following few paragraphs will discuss 
        feasibility of Context Sensitive Intraprocedural Analysis, 
        because call sites are not always known, 
        practical consequences and usual approaches to 
        make Context Sensitive Intraprocedural Analysis 
        more scalable.}
    
    % ----------------------------------------------------------------------------
    \section{Control Flow for Phalanger Approach}
        \note{Description of the analysis used in our case using the terminology 
        built up in the previous section. It will be more 
        abstract description, without technical details 
        about implementation.}
        
        \begin{itemize*}
            \item Modular approach that comes from assumption that 
                reasonably written PHP code will be close or equivalent to 
                statically typed code: (Aggressive Type Inference, by John Aycock: giving people 
                a dynamically-typed language does not mean that they write dynamically-typed programs.)
                \begin{itemize*}
                    \item No heap abstraction: we use classes -- enough for our purposes, no constant propagation, no null propagation, ...
                    \item No context sensitivity: we analyse routines in modular way.
                    \item Simple points-to analysis and lambda capture by ref.
                    \item Arrays: modelled as local variables.
                \end{itemize*}
        \end{itemize*}
        

\chapter{Related Work}

    Static analysis is a topic that is being actively researched. 
    A brief overview of the static analysis methods was presented 
    in the previous \wchapter{}. Control Flow for Phalanger's aim is 
    to adapt and apply some of the results of the research deemed 
    to be useful for the purposes of type analysis of PHP. 
    In related work, we focus on tools that also use static 
    analysis to analyse PHP code for different purposes. 
    We also shortly mention interesting tools for other 
    dynamic languages.

    \section{Security Vulnerabilities in Web Applications}
    
    Most of the existing work on static analysis of PHP is 
    focused on discovery of security vulnerabilities in 
    web applications that typically come from improperly 
    handled user input, also called taint-style vulnerabilities. 
    It is important for such analyses to be able to follow 
    the flow of data from global variables like \code{\$\_POST} 
    that contain user input, therefore more precise model of 
    heap memory is required so that flow of data in between 
    object instances and routine calls can be analysed. 
    
    An analysis for security vulnerabilities has also a different 
    model of usage. Such analysis can be run not so frequently, 
    for example, only before release or as a part of a continuous 
    build process. Interactive on-the-fly analysis in an 
    integrated development environment could also be a viable 
    use case, but typically not the main goal. Moreover, 
    such analysis is not likely to be run every time the 
    application is to be compiled or interpreted.
    
    Some of the available tools for detecting taint-style 
    vulnerabilities in PHP are Pixy\cite{jovanovic2006pixy} 
    and recently released Weverca: Web Verification Tool\cite{hauzarhunting}.    
    
    \subsection{Weverca: Web Verification Tool}
    
    Weverca is also based on the Phalanger parser, but it uses an older 
    version, therefore it does not support some of the newer language 
    constructs.
    
    Opposed to our modular approach, Weverca uses heap abstractions, 
    which in correspondence with the different goals of both projects 
    where Weverca focuses on tracking data that originated from user 
    input down to their usage.
    
    In order to make the interprocedural analysis context sensitive, 
    Weverca uses routines inlining: the whole routines body is 
    copied at the place of its invocation and formal parameters 
    are substituted for the actual parameters of the routine 
    invocation.
    
    % TODO: what do they do if this cannot be done?

    \section{Type Inference}
    
    Type inference for dynamic languages is typically implemented 
    for the purposes of compilation or interpretation. A notable implementation 
    is type inference for PHP in Facebook's Hip Hop project \cite{zhao2012hiphop}, 
    which is a compiler from PHP to C++ and a custom intermediate language 
    that can be run in the Hip Hop virtual machine. Hip Hop performs type 
    analysis in order to find a single type for a variable, so it can treat 
    it as statically typed variable during compilation. However, if a single 
    type for a variable cannot be determined, Hip Hop does not analyse 
    the type information any further and fall backs to the dynamic typing.    
        
    There are implementations of type inference for other dynamic languages. 
    Ecstatic\cite{madsen2007ecstatic} is type inference for Ruby 
    implemented using control flow insensitive cartesian product algorithm. 
    Rubydust\cite{an2011dynamic} introduces a \emph{constraint based dynamic 
    type inference} that infers static types based on dynamic program 
    executions.

    \subsection{Phantm}
    
    Phantm\cite{kneuss2010phantm} is a tool for detection of type related 
    errors. From all the projects mentioned in this chapter, the aim of 
    Phantm is closest to our project, which is why we also used Phantm 
    for evaluation and compared its results with ours in 
    section \ref{phantmresults} Comparison to Phantm.
    
    Phantm uses semi-dynamic and semi-static analysis approach. The web 
    application in question is run up to a defined point, which is invocation 
    of special Phantm's function that collects data about the state of the application, 
    especially, values of global variables. This data is then used as an initial 
    state for static analysis. The dynamic part of the analysis is called bootstrapping. 
    This design illustrates that although type related errors can be searched for 
    in generic frameworks, libraries or, for example, command line utilities 
    written in PHP, Phantm's focus is on complete web applications.
    
    %\note{Precise memory model, Routines inlining, ...}

    
    
    
    
    
    

\chapter{Design and Implementation}

    \section{Introduction}
    
    In this chapter we provide an overview of the design and 
    implementation of the type analysis described in 
    section \nameref{phalapproach}. Aside the type analysis 
    itself, the aim of Control Flow for Phalanger is to provide 
    a generic framework for implementation of any kind 
    of analysis, especially \emph{data-flow} based analyses.
    
    Control Flow for Phalanger includes implementation of 
    aforementioned type analysis, constant propagation analysis 
    and dead code detection. The results of those analyses 
    are made available to other user defined analyses 
    through a public interface, so that user defined analyses 
    can benefit from, for example, a resolved method call 
    that depends on the type of the variable the method 
    is called upon. Moreover, the analyses included in 
    Control Flow for Phalanger can report errors and warnings 
    discovered during analysing the code like, for instance, 
    dead code or type mismatches with documentation.
    
    Control Flow for Phalanger can also be used as a 
    library by other tools. For example, integrated 
    development environment plug-in can visualize the 
    issues reported by the default analyses, or compiler 
    can use the public interface for accessing the 
    analyses results to emit more efficient code.
    
    In the next section we discuss the implementation 
    constraints that come from the requirement to use 
    Phalanger front-end and to integrate Control Flow 
    for Phalanger with the whole Phalanger project -- 
    it should be designed so that it can be plugged 
    in between the Phalanger's front-end and back-end 
    and it should also provide public interface useful 
    for the Phalanger PHP Visual Studio tools.
    
    Section \nameref{overalldesign} provides an overview 
    of the architecture of Control Flow for Phalanger in 
    terms of high level modules and their interaction. 
    Finally, the last section of this chapter \nameref{impl} describes 
    some of the interesting bits of the implementation.
    
    \section{Implementation Constraints}
    
    \subsubsection*{Abstract Syntax Tree}     
    Phalanger front-end parses PHP code into an Abstract 
    Syntax Tree (AST) \cite{aho1985compilers} structure. 
    An example of such AST structure is depicted in 
    figure \ref{estexample}. This structure is then traversed 
    by the Phalanger back-end, which emits the 
    corresponding Microsoft Intermediate Language 
    (MSIL) instructions. Phalanger does not use any other 
    intermediate representation than AST and so the back-end 
    transforms AST directly to MSIL.
    
\begin{figure}[h]  
  \centering
    \includegraphics*[scale=0.5]{graphs/evaltree-ast.png}  
    \caption{Abstract Syntax Tree of code \code{(foo()+\$a)/\$b}.\label{estexample}}
\end{figure}      
    
    The classes that represent the AST nodes support extensible 
    attributes through which one can attach any additional 
    information to the nodes, or in other words, ``annotate'' the nodes. 
    This mechanism is used by Phalanger back-end and 
    Control Flow for Phalanger also uses the extensible 
    attributes to provide the results 
    of its analyses as discussed in the following section.
    
    \subsubsection*{Integrated Development Environment Integration}
    The PHP Tools for Visual Studio use Phalanger 
    front-end in order to parse PHP code into AST 
    and then the AST is again traversed to provide 
    code completion and other features. The AST nodes 
    hold necessary pieces of information, for example, 
    the position in the source file or documentation comments.
    Control Flow for Phalanger can seamlessly integrate 
    into this schema, because it provides its results 
    as annotations of the original AST nodes.
    
    The longer term aim of this project, not in the scope 
    of this \wthesis{}, is to replace the existing 
    algorithms for code completion, ``jump to definition'' 
    and ``find usages'' features. Because 
    with a dynamic language like PHP, it is not trivial to 
    find all the usages of, for example, a class or determine 
    a definition of, for instance, a field accessed 
    on some local variable. In order to provide more 
    precise results, type analysis is needed.
    
    \subsubsection*{Large PHP Projects}
    The implementation of the analysis should be able to handle 
    PHP projects with few thousands of files on a typical development 
    PC configuration under 1 minute and with less than 2GB of memory, 
    so that it can be integrated into the development process as a 
    part of a compiler or IDE plugin.
    
    \section{Overall Design}
    \label{overalldesign}
    
    The project is divided into several modules.
    \begin{itemize*}
        \item Control Flow Graph,
        \item Intermediate Representation of PHP Code (Phil, RPhil),
        \item Generic Data Flow Analysis Framework, 
        \item Tables with Type and Other Information, 
        \item Concrete Analyses:
        \begin{itemize*}
            \item Dead Code Elimination, 
            \item Aliasing Analysis, 
            \item Constant Propagation,
            \item Type Analysis.
        \end{itemize*}
    \end{itemize*}
    
    The interactions between those modules on a conceptual level 
    when performing an analysis are depicted in diagram \ref{overalldiagram}. 
    Green elements represent extension points. Red arrows represent the 
    core flow of the algorithm.

\begin{figure}[h]  
  \centering
    %\includegraphics*[width=\textwidth,height=\textheight,keepaspectratio,viewport=0 55 532 590]{img/ControlFlowModules.pdf}  
    \includegraphics*[width=\textwidth,height=\textheight,keepaspectratio,viewport=0 15 565 590]{img/ControlFlowModules.pdf}  
    \caption{Control Flow for Phalanger Design Overview\label{overalldiagram}}
\end{figure}

    In case of analysis of a single routine, the PHP source code is first 
    parsed by Phalanger front-end, then we construct a control 
    flow graph of the routine and start the Data-flow analysis. 
    Each control flow graph node, called basic block, contains AST 
    elements representing statements that are always executed 
    sequentially. The Data-flow analysis module directs the 
    generic algorithm of computing the data-flow equations, 
    but when it needs to perform a \emph{transfer function} or 
    an operation with a \emph{data-flow} value, it invokes the implementation of 
    those provided by the ``Concrete Data Flow Analysis''. 
    In order to make the implementation of \emph{transfer functions} 
    easier, the the list of basic block's AST elements is 
    transformed into an intermediate language, which is simplified 
    version of the AST elements and also includes some additional 
    information useful for analysis implementation.
    
    The analysis results can be saved as annotations of 
    corresponding AST nodes or they can be saved into the 
    \emph{global symbols} database that provides type and 
    other useful information about global symbols like 
    routines, classes, and others. Conversely, the 
    \emph{global symbols} database can be used by the 
    analysis to query the global symbols information.
    
    The following paragraphs describe some of the 
    modules in detail.

    \subsection{Intermediate Representation}
    The purpose of the intermediate representations is 
    to aid the design of the concrete analyses by 
    providing a simpler interface than full AST.
    However, at the same time, it was desirable to stay as 
    close to the original AST elements as possible in order 
    to easily propagate the results and to easily integrate 
    Control Flow for Phalanger into the Phalanger project.         
    Lastly the intermediate language should stay close enough 
    to PHP code so that PHP specific patterns can be recognized.    
    
    Control Flow for Phalanger uses two intermediate representations 
    on a conceptual level, however, they are usually not explicitly 
    constructed, but transformed from AST on the fly.
    
    \emph{Phil} stands for \emph{PHP Intermediate Language} and is 
    very abstract representations close to the original AST. 
    The main aim of Phil is to provide a framework for 
    traversing the AST elements in the order of their 
    execution in the smallest execution steps possible 
    with respect to their possible effects to the environment.     
    
    Phil contains only statements and expressions that can be 
    in a Basic Block, therefore it does not contain most of the 
    control flow changing statements like if, switch, or loops. 
    A Phil statement represents the smallest single step 
    of an execution that can change state of variables 
    or global state or throw an exception. Syntactic 
    constructs like \code{\$i++} are unfolded to 
    \code{\$i=\$i+1}, which is then split to 
    evaluation of binary expression and assignment expression 
    that uses the result of the binary expression. In some sense, 
    Phil can be viewed as a three address code.
    
    \emph{RPhil} stands for \emph{Resolved PHP Intermediate Language}. 
    RPhil is basically a Phil with resolved symbols where possible. 
    In order to resolve the symbols, the module building RPhil 
    can use names explicitly expressed in the code, for example, 
    for direct local variable access \code{\$a}, or results of 
    an analysis, for instance, results of the type analysis to 
    resolve method calls and fields references. Those results 
    are obtained from AST annotations and from \emph{global symbols}.
    RPhil itself is typically consumed by the analyses, 
    so the accuracy of RPhil and subsequently of the 
    analyses results can be improved by iterative execution 
    of the analyses.
    
    \subsection{Global Symbols}
    Global symbols database provides interface for 
    obtaining type information of global symbols, 
    namely routines, object fields, static fields, constants 
    and global variables. Definitions of those symbols can be 
    part of the analysed code or they can, for example, refer 
    to a built-in functions or external libraries. 
    For this reason, there are two interfaces 
    for the Global symbols database. 
    
    \emph{Code Tables} is a read and write interface for 
    a database of symbols that are part of the analysed code.     
    In this case the database can provide more than just type 
    information. Especially, we have a source code of definitions 
    of those symbols and so we can perform analysis in order to 
    infer their type information and save it back into 
    the \emph{Code Tables}.
    
    \emph{Type Tables} on the other hand provide only read
    access to only type information and few other important 
    pieces of information, concretely which parameters are 
    passed by reference, if a routine returns a reference 
    and similar. \emph{Type Tables} merge the data from 
    \emph{Code Tables}, which provide even more information 
    than needed, and any other user defined data source, 
    which can be, for example, a database with built-in symbols.    
        
    \subsection{Extensibility}
    The whole project is designed as a class library and 
    framework with many extension points. Some of the 
    functionality can be used independently. For example, 
    Control Flow Graph builder to generate diagrams.
    
    Nonetheless, in order to provide better usability, 
    Control Flow for Phalanger also contains 
    a simple Facade interface that plugs in together all 
    the necessary objects needed to perform a defined 
    analysis of a file or a given piece of code.
    
    \section{Implementation}
    \label{impl}
    This section describes some of the interesting parts of the 
    implementation in more detail.
    
    \subsection{Intermediate Languages Construction}
        We denote one statement of the intermediate language as
        a Phil or RPhil element. RPhil elements correspond 
        one-to-one to Phil elements, they only carry more 
        information. The aim of the implementation was to 
        save resources by not constructing the Phil (RPhil) 
        representation for every AST element explicitly and 
        saving it in memory.
        
        Under closer look, most of the AST elements correspond to 
        exactly one Phil element, or they are ignored, 
        which is the case of most control flow changing statements 
        and declaration statements. Those AST elements are used as 
        Phil elements as they are. However, elements that have to be 
        unfolded into several operations, such as \code{IncDecEx} 
        representing post or pre increment or decrement, have to 
        be represented by more elements. 
        
        One option would be to create a whole new alternative 
        AST structure that would correspond the unfolded code. 
        In our approach, we reuse the AST element to represent one 
        of the unfolded operations and create new special Phil 
        elements that wrap the original AST element and their 
        only purpose is to indicate the stage of the compound 
        operation. So for example, \code{IncDecEx} is broken down 
        to 
        \begin{itemize*}
            \item \code{Expression} that represents the variable access and 
                is recursively broken down to Phil elements, 
            \item \code{IncDecEx} that represents the binary plus operation,
            \item \code{IncDecPhilAssignment} that represents the assignment 
                and is a special additional Phil element.
        \end{itemize*}
        
        Phil elements are processed on the fly as they are transformed from 
        the original AST, therefore we do not have to explicitly save them 
        in memory. This means that every time a piece of AST is to be processed, 
        the transformation to Phil take place again, but in a fact, it is 
        a very light process. Nonetheless, we have an object instance to 
        represent every Phil element, so we do not have to process them 
        on the fly, they can also be saved into an array, which enables 
        us to perform the traversing of original AST only once, shall 
        it be a performance issue in the future.
        
        It is also possible to crate accurate control flow 
        graphs when it comes to try-catch blocks, because a single AST element, 
        can cause exception in any stage of its evaluation, therefore it should 
        be split into single execution steps and those put into separate 
        basic blocks so that an edge from each of them to the 
        catch block can be created. Note that the design is prepared 
        for this case, but it is not implemented.
    
    \subsection{Control Flow Graph}        
        
        \subsubsection*{Exceptions}
        For try code blocks every statement is placed in 
        its own basic block that is connected to the basic block 
        of the consecutive statement and connected to all 
        the possible catch code blocks.
        
        The possible catch blocks are chosen pessimistically, so 
        a statement in try code block is connected to all the 
        catch code blocks, even nested ones, up to a catch block 
        that catches generic \code{Exception} or to the 
        graph's \emph{exit} basic block. The picture \ref{cfgthrow} show 
        a control flow graph for a function with try-catch blocks.
        
\begin{table}[h]
  \begin{tabular}{ l | m{6cm} }
  \centering
    \includegraphics[scale=0.7]{src/throw.png}
  &
 
\begin{minipage}{6cm}
%pygmentize_begin php
% <?php
%    try {
%        foo();
%        try {
%            boo();
%        } catch (MyException $e2) {
%            handle($e2);
%        }
%    } catch (Exception $e) {
%        echo $e->message;
%    }
%    
%    echo 'exit';
%pygmentize_end
\begin{Verbatim}[commandchars=\\\{\}]
\PY{x}{ }\PY{c+cp}{\PYZlt{}?php}
    \PY{k}{try} \PY{p}{\PYZob{}}
        \PY{n+nx}{foo}\PY{p}{();}
        \PY{k}{try} \PY{p}{\PYZob{}}
            \PY{n+nx}{boo}\PY{p}{();}
        \PY{p}{\PYZcb{}} \PY{k}{catch} \PY{p}{(}\PY{n+nx}{MyException} \PY{n+nv}{\PYZdl{}e2}\PY{p}{)} \PY{p}{\PYZob{}}
            \PY{n+nx}{handle}\PY{p}{(}\PY{n+nv}{\PYZdl{}e2}\PY{p}{);}
        \PY{p}{\PYZcb{}}
    \PY{p}{\PYZcb{}} \PY{k}{catch} \PY{p}{(}\PY{n+nx}{Exception} \PY{n+nv}{\PYZdl{}e}\PY{p}{)} \PY{p}{\PYZob{}}
        \PY{k}{echo} \PY{n+nv}{\PYZdl{}e}\PY{o}{\PYZhy{}\PYZgt{}}\PY{n+na}{message}\PY{p}{;}
    \PY{p}{\PYZcb{}}
    
    \PY{k}{echo} \PY{l+s+s1}{\PYZsq{}exit\PYZsq{}}\PY{p}{;}
\end{Verbatim}
\end{minipage}

  \\
  \end{tabular}
  \caption{Control Flow Graph with Exceptions\label{cfgthrow}}  
\end{table}
        
        \subsubsection*{Edges}
        Control Flow Graph edges can have an optional attribute which 
        states the expression that has to hold if this edge is taken 
        during the execution. For example, an edge to then branch of 
        \code{if (\$x==3)} will have expression \code{\$x==3} and the 
        analyses may work out from it that \code{\$x} is equal to 
        \code{3} in the basic block corresponding to the then branch.
        
    \subsection{Data Flow Analysis}
        A data flow analysis (DFA) in general can be performed on any graph, 
        and so even in our case, we did not want it to be tied 
        Control Flow Graph. For this purpose, our implementation of DFA is 
        performed on interfaces that Control Flow Graph implements, 
        but they can be implemented, for example, by a definition-use 
        graph\cite{aho1985compilers} and DFA can be run on this graph too.
        The interfaces are depicted in figure \ref{graphifaces}.
        
\begin{figure}[h]  
  \centering
    \includegraphics*[width=\textwidth,height=\textheight,keepaspectratio]{img/graph-ifaces.png}  
    \caption{Generic Graph Interfaces for DFA\label{graphifaces}}
\end{figure}    

        The generic data flow framework handles the order in which the 
        graph nodes should be visited, compares the input and output 
        data flows and decides when the analysis has reached a fix-point. 
        However the concrete type of data flow, operations with data 
        flow instances and the transfer function are left to be defined 
        by a concrete analysis. In Control Flow for 
        Phalanger, two types of analysis can implemented. The basic one 
        processes only nodes; the ``branching'' one also processes edges, 
        in which case it can take the branching expressions of Control 
        Flow Graph into account, for example, but in general any information 
        that the concrete implementation of \code{IEdge} can provide.
        
        The operations with data flow objects could be carried out by the 
        objects themselves, but this would mean that already existing 
        classes that happen to be suitable for being a data flow would 
        have to be wrapped. And also one data flow representation, could 
        not have different operations for different analyses. An example 
        of this is \code{BitVector} class from .NET class library: it 
        cannot implement any additional interface, and some analysis 
        perform union of two vectors as the meet operation, while 
        others perform intersection.
        
        Nonetheless, having the data flow objects implement the operations by 
        themselves is more convenient and allows better encapsulation. 
        Because Control Flow for Phalanger is meant as a framework for as 
        well as software on its own, both scenarios are supported and 
        some convenient generics based implementations of required 
        interfaces are provided. The whole design is captured in 
        diagram \ref{dataflowifaces}.
        
\begin{figure}[h]  
  \centering
    \includegraphics*[width=\textwidth,height=\textheight,keepaspectratio]{img/dataflow-ifaces.png}  
    \caption{Interfaces for concrete Data Flow Analyses\label{dataflowifaces}}
\end{figure}

        A simple example of concrete Data Flow Analysis is the built-in 
        constant propagation analysis, which is discussed in the 
        following subsection.        
    
    \subsection{Analyses}
        \subsubsection*{Dead Code Elimination}
        
        The Dead Code Elimination is based on the Reverse Post Order 
        algorithm, which is supposed to order nodes in a way that 
        is the most beneficial for Data Flow Analysis and has to be 
        done anyway in order to perform Data Flow Analysis.
        
        Note that the Reverse Post Order algorithm as a part 
        of the Data Flow Analysis module is implemented 
        in a generic way for any \code{IGraph} implementations. 
        
        The algorithm performs a graph search from the \emph{Start} 
        node and after it finishes, the unvisited nodes are 
        at the end of the list with all the nodes. At this point, 
        the list with all the nodes is traversed from the last 
        element and the unvisited nodes are removed until a 
        first visited node is reached.
        
        The Control Flow Graph design allows to mark 
        some edges as ``not executable'' if the branching 
        condition is always false. Such edges will be ignored 
        when the Dead Code Elimination is performed, however, 
        the tagging of edges with false branching condition 
        has not been implemented in the final version.
        
        \subsubsection*{Constant Propagation}
        Constant Propagation represents a simple example of non-branched 
        implementation of a concrete Data Flow Analysis. 
        
        The lattice for the data flow values of single variable is 
        depicted in figure \ref{constlattice}. The Data Flow type is class 
        \code{ConstantPropagationDataFlow}, which wraps an array 
        that contains the value of each variable. 
        The value is an \code{object} instance, so it can be \code{null}, which 
        is the least element of the lattice, or its value can be concrete 
        singleton object instance that by convention represents 
        \code{NotAConstant}, which is the greatest element of the lattice.
        
\begin{figure}[h]  
  \centering
    \includegraphics{img/ConstLattice.pdf}  
    \caption{The Lattice for Constant Propagation\label{constlattice}}
\end{figure}        
        
        The transfer function annotates every expression 
        with its constant value if possible. An access to a 
        variable is annotated with this variable's value 
        from the data flow, some expressions can be symbolically 
        executed using the annotations to get operands values, 
        and constants are annotated with their value retrieved 
        from Type Tables.
        
        The data flow is updated only for assignment statement, 
        reference assignment statement and for function call, 
        because some local variables can be passed by reference, 
        which can change their value. The right hand side of 
        the assignment is an expression, which should already be 
        annotated with its value that is used for the data 
        flow value.

    \subsection{Type Information Representation}        
        The theoretical side of workings of the type analysis was 
        discussed in \wsection{} \nameref{phalapproach}.         
        In this paragraph we discuss representation of 
        type information for a single variable or a single expression. 
        The representation is based on the type information 
        lattice described earlier. It is a set of possible 
        basic types, class types and \code{object}. It can 
        have flags ``subclasses'' and ``type hint''. 
        There are few things to note:
        \begin{itemize*}
            \item \code{object} represents any object type, 
            \item \emph{Any} is a special \emph{data-flow} value that 
                represents any possible type, 
            \item empty set $\emptyset$  is a special \emph{data-flow} value that 
                represents uninitialized type.
        \end{itemize*}
        
        The aim is to represent the type information efficiently, 
        because we want to annotate every expression with type 
        information and because we want a memory efficient 
        representation of the \emph{data-flow}, which is an 
        array of type information for all variables.
        
        The type information is always tied to a concrete routine: 
        it is either annotation of one of the expressions in 
        the routine's body, or type of one of the local variables. 
        For every routine we create a special ``context'' object 
        with a list of all referenced types $\mathbb{T}$. This way 
        every type has a unique index. Let us denote the index 
        of type $K$ as $\mathit{index}(\mathbb{T}, K)$ the 
        type under index $i$ as $\mathit{type}(\mathbb{T}, i)$.
        Furthermore, we made an assumption that a single routine 
        is unlikely to reference more than $64$ distinct types. 
        From this follows that we can represent a subset of 
        $\mathbb{T}$ as a 64 bit integer, where bit with index 
        $i$ indicates whether the type $\mathit{type}(\mathbb{T}, i)$ 
        is present in the set or not. With this representation, 
        we can implement very efficient set operations with 
        bitwise operators.
        
        However, this representation does not reflect the 
        lattice we described earlier. We shuffle 
        the type indexes to the left and reserve first few 
        indexes from right for special bit flags:
        
        \begin{itemize*}
            \item\textbf{any flag} if present, other bits should be ignored and the whole 
                type information value is deemed as the lattice element \emph{Any}.            
            \item\textbf{object flag} if present any class types should be ignored.
            \item\textbf{type hint flag}
            \item\textbf{subclasses flag}
        \end{itemize*}
        
        From the properties of bitwise or, it follows that 
        a bitwise or (union) of two type information instances 
        represented with this schema, will give us their 
        lowest upper bound. Let us, for example, consider 
        a union of two type information instances where one 
        has \textbf{any flag} set. The union will have 
        also \textbf{any flag} set, which in the lattice 
        corresponds to $\emph{Any}\wedge{}x=\emph{Any}$ 
        for every $x$. The same behaviour works with 
        respect to the other special flags.
        
        Last thing we need to deal with is when a routine references 
        more types than the number we can represent. In such case, 
        any type whose index would be greater than $63$, is treated as 
        if its index was exactly $63$. This means that types with 
        index greater or equal to $63$ will share one bit and 
        therefore we loose precision, because we cannot distinguish those 
        types from each other. However, it is a safe approximation and 
        a price we pay for the memory efficiency.

%pygmentize_options: -O startinline=True
\chapter{Results}


\section{Comparison to Phantm\label{phantmresults}}



\section{Evaluation on open source PHP code}

The project has been evaluated on the following open source PHP frameworks and 
websites.

\begin{description}
    \item[PHPUnit:] a port of JUnit unit testing framework for PHP; it is a mature 
    and well established project that has been developed for more than 6 years 
    by 156 contributors. Being a unit testing framework, PHPUnit itself has extensive 
    unit test suite. For the experiment the master branch of the clone of the 
    repository retrieved on 18.5.2014 has been used. 
    
    \item[Zend Framework:] popular general purpose PHP framework. Lorem ipsum...
        \note{over 3000 problems reported. Quite a big bite to chew with analysing and categorizing all of them.}
        
    \item[Nette:] another popular PHP framework for building websites.
        \note{Findings not so impressive: 1 (probably only) documentation error :(}
        
    \item[Piwik:] open source version of Google Analytics. \note{few hundred problems found, not categorized yet}
    
    \item[PrestaShop:] open source e-commerce solution. \note{few hundred problems found, not categorized yet}
    
    \item[Composer:] popular dependency management system for PHP libraries. \note{few hundred problems found, not categorized yet}
\end{description}

An evaluation always started with downloading a git repository with the 
latest source code of given PHP project. Then the analysis was run and 
all the discovered problems were collected and categorized.
Actual errors were rectified and recorded as commits in the 
git repository. 

Often one real issue in the source code caused several 
warnings to be reported by the tool. For instance, if a documentation 
of a type of a field was not correct, most the lines where 
a value was assigned to that field were reported, however, the 
root of the cause was actually the only one line with the 
wrong type documentation. Such cases were counted as a 
single problem.

\subsection{Problems Taxonomy}

The problems were divided into few main categories 
described below. Some of the problems recurring 
across all the projects are discussed in this section, 
while the more project specific problems are 
investigated later each project's section.

\paragraph*{Style:} a category of problems that 
are not directly affecting the functionality of the 
application, but they can be considered as a bad 
coding style. 

\begin{itemize}
    \item[] \textit{Relaying on default return value} -- when an execution of a 
        routine does not end with a return statement, its 
        return value is \code{NULL} by default, which can then 
        be cast to values of other types, like \code{false} 
        for example. Developers rely on this feature and omit 
        the return statement if they want to return \code{NULL} or 
        something that \code{NULL} can be cast to.
\end{itemize}

\paragraph*{Documentation:} inconsistency of the PHPDoc type 
documentation and what the the code does. This category includes 
only cases where it is clear that the documentation is wrong, not the code, 
for example, due to updates in the code that were not reflected 
in the documentation. The inconsistency may also indicate 
another problem, in which case it does not belong to this category. 

Interestingly, most of the inconsistencies of type of a parameter 
of a function call typically lead through several routines that 
only forward the parameter to the next routine until 
eventually the routine that has a wrong documentation is reached.

\begin{itemize}
    \item[] \textit{Missing} \code{false} \textit{in return value type documentation} -- 
        this is common pattern in PHP where a routine returns \code{false} 
        when it fails to do what it was supposed to do. For example, 
        function \code{fopen} returns \code{resource} of \code{false} 
        if the resource could not be opened. It is so common that 
        developers tend to forget to put \code{false} into the documentation.
\end{itemize}

\paragraph*{Actual Error:} includes all problems that can cause 
an unexpected exception or unexpected runtime error or notice.


\paragraph*{False Positive:} problems reported by the tool that 
are not in fact real problems. Includes only false positives 
that cannot be eliminated because of fundamental design reasons 
or are not intended to be eliminated, because it would 
cause some actual positives to be missed.


\begin{itemize}
    \item[] \textit{Unused routine arguments} -- when a method is an override of 
        some base method, it can have the same signature and if some of the 
        parameters are not used, they are not reported. There are however 
        some cases where the routine is implementing some interface by convention 
        that is not explicitly expressed in the syntax of PHP.
        For example, the pre-object-oriented pattern for global functions 
        overriding. In such case the analyzer cannot determine that the unused 
        parameter is in fact a part of an interface. Note that such function 
        could omit the unused parameter and everything would work the same, 
        therefore this may or may not be considered a false positive.
    \item[] \textit{Amendable false positive} -- false positives 
        that are reported, although the algorithm the analyzer 
        is using should not report them. Those indicate errors 
        in the implementation.
    \item[] \textit{Built-in documentation errors} -- false positives 
        due to the inaccuracy of the documentation of built-in 
        functions and classes that was used in the experiment.
\end{itemize}

\subsection{Summary}

The following table provides a summary of the problems found. 
There is a table that lists concrete problems found PHPUnit 
available in appendix.

\newcommand{\subcat}[1]{\hspace{0.5cm}\small{\textit{#1}}} 
\newcommand{\reldefret}{\subcat{default return value}}

\newcommand{\sumh}[1]{\textbf{#1}}
%\newcommand{\sumh}[1]{\begin{turn}{60}#1\end{turn}}

\begin{center}
    \begin{tabular}{| p{5cm} | r | r | r | r | r |}
    \hline
    \sumh{Category}         &   \sumh{PHPUnit}      &   \sumh{Zend}       &   \sumh{Nette}    &   \sumh{WP}    &   \sumh{total}   \\ \hline
    Style                   &   6                   &       NA                      &   NA              &   NA                  &   6       \\ \hline
    \reldefret              &   2                   &       NA                      &   NA              &   NA                  &   2       \\ \hline        
    Documentation           &   10                  &       NA                      &   NA              &   NA                  &   10      \\ \hline    
    \subcat{missing false}  &   3                   &       NA                      &   NA              &   NA                  &   3       \\ \hline
    Actual Error            &   1                   &       NA                      &   NA              &   NA                  &   1      \\ \hline    
    False positive          &   8                   &       NA                      &   NA              &   NA                  &   8      \\ \hline    
   \subcat{unused arguments}&   1                   &       NA                      &   NA              &   NA                  &   1      \\ \hline        
    \subcat{amendable}      &   4                   &       NA                      &   NA              &   NA                  &   4      \\ \hline            
 \subcat{built-in doc error}&   3                   &       NA                      &   NA              &   NA                  &   3      \\ \hline
    \textbf{Total} 
    (excl. false positives) &17                   &       NA                      &   NA              &   NA                  &   17      \\ \hline                
    \end{tabular}
\end{center}



\subsection{Selected Problems}

\subsubsection*{Actual Error When Handling \code{DOMElements}}

The error is related to the following code (shortened).

%pygmentize_begin php
% function assertEqualXMLStructure(
%   DOMElement $expectedElement/*, ...*/) {
%   ///...
%   PHPUnit_Util_XML::removeCharacterDataNodes($expectedElement);
%   PHPUnit_Util_XML::removeCharacterDataNodes($actualElement);
%   //...
%   for ($i = 0; $i < $expectedElement->childNodes->length; $i++) {
%       self::assertEqualXMLStructure(
%           $expectedElement->childNodes->item($i) /*<<< error */
%           /*...*/);
% }
%pygmentize_end
\begin{Verbatim}[commandchars=\\\{\}]
 \PY{k}{function} \PY{n+nf}{assertEqualXMLStructure}\PY{p}{(}
   \PY{n+nx}{DOMElement} \PY{n+nv}{\PYZdl{}expectedElement}\PY{c+cm}{/*, ...*/}\PY{p}{)} \PY{p}{\PYZob{}}
   \PY{c+c1}{///...}
   \PY{n+nx}{PHPUnit\PYZus{}Util\PYZus{}XML}\PY{o}{::}\PY{n+na}{removeCharacterDataNodes}\PY{p}{(}\PY{n+nv}{\PYZdl{}expectedElement}\PY{p}{);}
   \PY{n+nx}{PHPUnit\PYZus{}Util\PYZus{}XML}\PY{o}{::}\PY{n+na}{removeCharacterDataNodes}\PY{p}{(}\PY{n+nv}{\PYZdl{}actualElement}\PY{p}{);}
   \PY{c+c1}{//...}
   \PY{k}{for} \PY{p}{(}\PY{n+nv}{\PYZdl{}i} \PY{o}{=} \PY{l+m+mi}{0}\PY{p}{;} \PY{n+nv}{\PYZdl{}i} \PY{o}{\PYZlt{}} \PY{n+nv}{\PYZdl{}expectedElement}\PY{o}{\PYZhy{}\PYZgt{}}\PY{n+na}{childNodes}\PY{o}{\PYZhy{}\PYZgt{}}\PY{n+na}{length}\PY{p}{;} \PY{n+nv}{\PYZdl{}i}\PY{o}{++}\PY{p}{)} \PY{p}{\PYZob{}}
       \PY{n+nx}{self}\PY{o}{::}\PY{n+na}{assertEqualXMLStructure}\PY{p}{(}
           \PY{n+nv}{\PYZdl{}expectedElement}\PY{o}{\PYZhy{}\PYZgt{}}\PY{n+na}{childNodes}\PY{o}{\PYZhy{}\PYZgt{}}\PY{n+na}{item}\PY{p}{(}\PY{n+nv}{\PYZdl{}i}\PY{p}{)} \PY{c+cm}{/*\PYZlt{}\PYZlt{}\PYZlt{} error */}
           \PY{c+cm}{/*...*/}\PY{p}{);}
 \PY{p}{\PYZcb{}}
\end{Verbatim}

The method \code{assertEqualXMLStructure} accepts only instances 
of \code{DOMElement}, but it invokes itself recursively with first 
argument of type \code{DOMNode}. Because according to the 
PHP documentation the value of the \code{childNodes} property 
of interface \code{DOMNode} is an instance of \code{DOMNodeList} 
and the method \code{item(integer)} of \code{DOMNodeList} 
returns \code{DOMNode}, it is a type mismatch error as \code{DOMNode} 
is not subtype of \code{DOMElement}.

In the PHP implementation of DOM model, the only implementations of 
\code{DOMNode} either inherit from \code{DOMElement} or 
implement \code{DOMCharacterData}, and those are removed 
from the \code{childNodes} collection by \code{removeCharacterDataNodes}. 
Therefore, in most cases, this code behaves as expected. 

However, the \code{DOMNode} interface can be implemented by any user 
defined class, which does not have to inherit from 
\code{DOMElement}, and if an instance of such class was present 
in the \code{childNodes} collection, the code would cause an 
exception when trying to invoke \code{assertEqualXMLStructure} 
with an argument of wrong type.

Note that if method \code{removeCharacterDataNodes} removed 
all the child nodes that are not instances of \code{DOMElement}, 
the code would be correct, but the error would still be reported, 
therefore it would be false positive.



\chapter{Conclusion}

    \section{Future Work}
    
    \note{arrays support.}
    
    \note{Integer interval analysis, make sure 
    the analysis can be conservative enough and integrate 
    it with the compiler back-end.}    
    
    \note{Spartial Vectors for ExTypeInfos.}
    
    \note{CFG-free analysis without backwards edges, 
    comparison to what we have.}
    
    \note{Another direction towards precision: implementing memory model.}


%%% Seznam použité literatury
\def\bibname{Bibliography}
\addcontentsline{toc}{chapter}{\bibname}
\bibliographystyle{ieeetr}
\bibliography{bibliography}

%%% Tabulky v diplomové práci, existují-li.
\chapwithtoc{List of Tables}

\begin{center}
    \begin{tabular}{| l | l | l | p{6cm} |}
    \hline
    File                             &   Line    &   Category       &   Note  \\ \hline
    \path{Framework\TestCase.php}    &   1722    &   Style          &   Foreach used in form \code{foreach(\$array as \$key=>\$val)} although the \code{\$key} variable was not used anywhere. \\ \hline
    \path{Framework\TestCase.php}    &   1726    &   Style          &   \\ \hline    
    \path{Framework\Assert.php}      &   1861    &   Actual Error   &   Discussion in the \wthesis. \\ \hline
    \path{Framework\Assert.php}      &   1960    &   Documentation  &   The routine code allows one of the arguments to be an \code{array} and works with it as such, but the documentation states that it can only be \code{boolean}. \\ \hline        
    \path{Framework\Assert.php}      &   1896    &   Documentation  &   \code{assertSelectEquals} method restricts its parameter type to be \code{integer} and is invoked with \code{boolean}. 
                                                                        However, the parameter value gets only forwarded to \code{convertSelectToTag}, which accepts any type (mixed). \\ \hline    
    \path{Util\XML.php}              &   544     &   Documentation  &   \multirow{2}{6cm}{Missing \code{false} in return value documentation.} \\ \cline{1-3}    
    \path{Util\Test.php}             &   294     &   Documentation  &   \\ \hline
    \path{Util\GlobalState.php}      &   351     &   False Positive &   \multirow{2}{6cm}{Built-in documentation error.} \\ \cline{1-3}    
    \path{Util\Test.php}             &   360     &   False Positive &   \\ \hline
    \path{Framework\TestCase.php}    &   1941    &   Documentation  &   Field \code{mockObjectGenerator} is documented to have type \code{array}, 
                                                                        but value of type \code{MockObject\_Generator} 
                                                                        is assigned to it. The documentation should be updated. \\ \hline
    \path{Util\GlobalState.php}      &   351     &   False Positive &   Built-in documentation error. \\ \hline
    \path{Util\Test.php}             &   46      &   False Positive &   Function \code{trait\_exists} is conditionally declared if it does not exist.
                                                                        It follows the same signature of actual built-in function \code{trait\_exists}, 
                                                                        but it has empty body, therefore it does not use its arguments. \\ \hline
    \end{tabular}
\end{center}
%Not finished!!!

\openright
\end{document}
