%pygmentize_options: -O startinline=True

\chapter{Analysing PHP Code}

    The PHP programming language first appeared in 1995\cite{phphist}. Over the years 
    the language has involved and so have the ways how programmers were using it. 
    This project focuses on PHP version 5.5\footnote{From this point, 
    if the PHP version is not stated explicitly, it is implicitly 5.5.} 
    and the aim for the analysis 
    is to work well on PHP source code written in an object oriented manner, 
    using modern PHP patterns and idioms that are described later in 
    this text. The analysis, however, should work reasonably good on 
    any valid PHP code. We do not focus only on websites, but also on 
    PHP libraries and frameworks that by themselves do not contain 
    any PHP files that produce HTML or any other output for the user.
    
    \section{PHP semantics}
    This section describes some important parts of the semantics of 
    the PHP programming language, especially those that represent a 
    challenge for static analysis.
    
    \subparagraph*{}    
    In PHP, local or global variables, object fields and function or 
    method parameters are dynamically typed, which means that they 
    can hold values of completely different types at different 
    times of execution.
    
    \subsection{Local Variables}
    Local variables in PHP do not need to be declared explicitly. 
    Instead the first usage of a variable is also its declaration. 
    If a variable's value is used before the variable got any 
    value assigned, then the interpreter generates a notice, 
    however the execution continues and value \code{null} is 
    used instead. A variable can get a value assigned to it when it 
    appears on a left hand side of an assignment or when a 
    reference to that variable is created, in which case it gets value 
    \code{null}. Note: references are discussed in one of the following 
    subsections.

    The scope of a local variable is always its parent function not the 
    parent code block as in other languages like C or Java. So in the following 
    example, the usage of variable \code{\$y} at the end of the function 
    can generate uninitialized variable notice, however, if \code{\$x} 
    was equal to \code{3}, \code{\$y} will have a value although it 
    was declared in the nested code block.

    %pygmentize_begin php
    % function foo($x) {
    %   if ($x == 3) { $y = 2; }
    %   echo $y; // uninitialized variable if x != 3
    % }
    %pygmentize_end
    
    \subsection{Global and Local Scope}

    \subsection{Closures}
    PHP also supports anonymous functions. An anonymous function has its 
    own scope as any other function and its local variables are not visible 
    to the scope where it was declared

    \subsection{Interesting Control Flow Structures}
    Arbitrary expressions in continue and break.
    
    \subsection{Conditional Declarations}
                
    \subsection{Auto-loading}
    
    \subsection{PHPDoc Annotations}
    
    \section{Static Code Analysis}
        More detailed description of what static analysis is 
        (as opposed to for example explicit model checking, 
        verification, etc.). Terminology: context sensitive, 
        path sensitive, symbolic execution, abstract interpretation, etc.    
    
        \subsection{Data Flow Analysis}
        \subsection{Rapid Analysis}
        \subsection{Abstract Domain}
