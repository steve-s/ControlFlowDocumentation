\chapter{Introduction}

    \section{The Problem}
    
    Three out of the top ten programming languages in TIOBE index\cite{tiobe} -- 
    an indicator of programming languages popularity, are dynamically typed languages. 
    One of the reasons for their popularity is that they can be easier to 
    use and suitable for fast prototyping.
    But at the same time the possibility to omit type information, which might 
    be helpful during the early stages of a software project, can lead to more 
    error prone code, and eventually to problems in later phases of the 
    development and maintenance. Dynamic typing is also challenging for the 
    compiler's or interpreter's designers. With the type information, 
    a compiler is usually able to emit more efficient code.
    
    Programmers are aware of the possible problems with the maintenance of 
    dynamically typed code and they often include the type information in 
    documentation comments. However, the correspondence of the documentation 
    and the actual code is not checked, and moreover the compiler usually 
    does not take any advantage of having type hints in the comments.
    
    The PHP programming language is one of the mentioned popular dynamic 
    languages and Phalanger \cite{benda2006phalanger} is an implementation 
    of a PHP compiler that compiles PHP code into the .NET intermediate code. 
    Phalanger was developed at the Department of Software Engineering 
    of the Charles University in Prague. A part of the 
    Phalanger project is also an implementation of PHP tools 
    for Visual Studio.

    Because of its dynamic nature, PHP code is more difficult to analyse 
    than code written in a statically typed language, especially if we want the 
    analysis to be reasonably fast so that it can be used 
    in everyday development.  There is ongoing research into the 
    static analysis methods for many different families of programming languages, 
    including dynamic languages. The problem this project is addressing 
    is to adapt and apply those methods on a real world and widely 
    used programming language PHP. The result is a library that is capable of 
    performing static analysis of PHP code and can be integrated into 
    the Phalanger project. The library should be capable of supporting many 
    kinds of analysis, for example constant propagation. However, the main goal 
    is to provide a type analysis in order to discover possible type 
    related errors and mismatches with the type information in the 
    documentation comments. Furthermore, in the future the library 
    can be integrated into the Phalanger as a middle-end to provide 
    optimizations for the compiler.

    This project and the library has a code name \emph{Control Flow for Phalanger} 
    and we refer to it using this name in the following text.
    
    \section{\wThesis{} structure}
    
    The following \wchapter{} describes the challenges connected with analysing source code 
    written in the PHP programming language, approaches to static analysis of 
    source code in general and our adoption of those to PHP. 
    In \wchapter{} 3, we briefly discuss existing software of 
    this kind. \wChapter{} 4 provides more detailed description of our implementation. 
    The analysis has been evaluated on several middle to large sized open source 
    PHP projects and the results are presented in \wchapter{} 5.
    