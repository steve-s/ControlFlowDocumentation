\chapter{Introduction}

    \section{Problem Description}
    
    Three out of top ten programming languages in TIOBE index\cite{tiobe} 
    are dynamically typed languages. One of the reasons for their popularity 
    is that they are usually easier to use and suitable for quick prototyping.
    But at the same time the possibility to omit type information, which might 
    be helpful during the early stages of a software project, can lead to more 
    error prone code, and eventually to problems in later phases of the 
    development and maintenance. Dynamic typing is also challenging for the 
    compilers or interpreters designers. With the type information, 
    a compiler is usually able to emit more efficient code.
    
    Programmers are aware of the possible problems with the maintenance of 
    dynamically typed code and they often document the type information in 
    documentation comments. However, the correspondence of the documentation 
    and the actual code is not checked, and moreover the compiler usually 
    does not take any advantage of having type hints in the comments.
    
    The PHP programming language is one of the mentioned popular dynamic 
    languages and Phalanger is an implementation of a PHP compiler 
    that compiles PHP code into the .NET intermediate code, which was 
    developed at the Department of Software Engineering 
    of the Charles University in Prague. A part of the 
    Phalanger project is also an implementation of PHP tools 
    for Visual Studio.

    \subparagraph*{}    
    Because of its dynamic nature, PHP code is more difficult to analyse 
    than code written in a statically typed language, especially if we want the 
    analysis to be reasonably fast so that it can be used 
    in everyday development.  There is an ongoing research of the 
    static analysis methods for many different families of programming languages, 
    including dynamic languages. The problem this project is addressing 
    is to adapt and apply those methods on a real world and widely 
    used programming language PHP. The result is a library that is capable of 
    performing a static analysis of PHP code and can be integrated into 
    the Phalanger project. The library should allow to plug in any 
    kind of analysis, for example constant propagation. However, the main goal 
    is to provide a type analysis in order to discover possible type 
    related errors and mismatches with the type information in the 
    documentation comments and possibly to allow the compiler to 
    emit more efficient code.    
    

    \section{Implementation Constraints}
    More detailed description: usage of Phalanger front end, have to 
    provide support for compiler back end and IDE (re-analysis of once analysed code), 
    focus on object oriented PHP 5 style projects.
    
    \question{Leave this here or put it in the chapter about implementation?}
    
    \section{Thesis structure}