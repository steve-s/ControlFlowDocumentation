\chapter{Implementation}

    \section{Implementation Specific Constraints}
    
    One of the requirements was that the project should be implemented 
    in the context of the Phalanger project. It should be ready 
    to be plugged in between the Phalanger's front-end and back-end and 
    it should also provide public interface useful for the 
    Phalanger PHP Visual Studio tools.
    
    \paragraph{Abstract Syntax Tree.} Phalanger front-end parses PHP code 
    into a Abstract Syntax Tree (AST) \cite{aho1985compilers} structure. 
    This structure is then traversed by the back-end using the 
    visitor design pattern \cite{gamma1994design}. Phalanger does not 
    use any other intermediate representation than AST.
    
    In order to reduce the memory consumption and provide better 
    modularity, Phalanger code went through small architecture 
    refactoring before this project was started. 
    
    The classes representing the AST nodes originally contained 
    the code and data needed for emitting the corresponding MSIL code. 
    That is, however, not necessary.
    What this coupling meant is wherever a programmer wanted 
    to use just the parser, she needed to reference the 
    back-end part of the compiler. Said in other words, the 
    front-end and back-end were not divided into separate units.
    
    \paragraph{Integrated Development Environment Integration.}
    The PHP Tools for Visual Studio use Phalanger front-end in order to 
    parse PHP code into AST and then the AST is again traversed to provide 
    code completion and other features. All the AST nodes hold necessary 
    pieces of information, e.g. the position in the source file or 
    documentation comments.
    
    The aim of this project is to provide the results of type analysis 
    and other analyses to the integrated development environment so 
    that the possible errors and warnings can be visualised, e.g. underlined.
    
    The longer term aim of this project, not in the scope of this thesis, 
    is to replace the existing algorithms for code completion, 
    ``jump to definition'' and ``find usages'' features. Because 
    with a dynamic language like PHP, it is not trivial to 
    find all the usages of e.g. a class or determine 
    a definition of e.g. a field accessed on a some local variable. 
    In order to provide more precise results, 
    type analysis is needed.
    
    One of the challenging parts of integrated development 
    environment integration is also dealing with incomplete code 
    that is being typed in by the user. Therefore one of the requirements 
    was also that the analysis should be capable of 
    performing an ad-hoc re-analysis of once analysed 
    code with a new statement added. This ad-hoc re-analysis 
    should be, if possible, more effective that doing the 
    whole analysis again.
    
    \section{Overall Design}
    The project can be divided into several parts:
    \begin{itemize}
        \item Control Flow Graph construction,
        \item Data Flow Analysis framework, 
        \item Dead Code Elimination,         
        \item Aliasing Analysis,
        \item Constant Propagation Analysis, 
        
    
    \section{Control Flow Graph and Data-flow Analysis}
        \begin{itemize}
            \item Control Flow graph construction
            \item discussion of the choice of intermediate representation (or in fact, why we do not use any intermediate representation).
            \item generic framework for Data Flow analyses
            \item support for re-analysing pieces of code
        \end{itemize}
    
    \section{Aliasing and Constant Propagation Analysis}
    
    \section{Tables}
        Dependencies handling. 
    
    \section{Type Analysis}        
        \begin{itemize}
            \item Type Information representation
            \item Data Flow representation
            \item AST annotations
        \end{itemize}
