\chapter{Conclusion}

    In this thesis we presented a project with code name 
    Control Flow for Phalanger, which can analyse PHP 
    source code in order to discover type related errors 
    and mismatches with type documentation.
    
    The Control Flow for Phalanger was evaluated on 
    three real world PHP project. Although, the tool 
    does not use heap abstraction and 
    does not perform context sensitive analysis, it was 
    still capable to discovered several real 
    issues with a good ratio of false positives. 
    
    This result may indicate that, were some imprecision 
    in the analysis results can be tolerated, the 
    modular approach we used can give results comparable 
    to those of tools that use more complex methods.

    \section{Future Work}
    
        The Control Flow for Phalanger has not yet been fully 
        integrated into the Phalanger project. This includes 
        integration with the compiler in order to enable 
        code optimizations and evaluation of the possible 
        performance gain when running PHP websites like WordPress.
    
        The analysis itself can be improved on several fronts.
        
        Array support has not been implemented yet. Variables 
        that can be of type array are analysed properly, but 
        the structure of the array is not analysed, therefore 
        any time an element of an array is accessed, we do not 
        have any type information for it and have to assume 
        the worst, that is it can be any type.
        
        Type information is represented using a 64 bit value, 
        however, we can go even further and represent the type 
        information with 8 bit number, which will be an index 
        into a table with all the possible type information 
        instances for one routine. This would give us 255 
        possible combinations of types, which we assume is 
        enough for most of the routines.
        