\chapter{Conclusion}
\label{conclusion}
    In this \wthesis{} we presented a project with code name 
    Control Flow for Phalanger, which can analyse PHP 
    source code in order to discover type related errors 
    and mismatches with type documentation. Control Flow 
    for Phalanger also provides a framework for 
    implementation of any other kind of analysis.
    
    The type analysis in Control Flow for Phalanger 
    is based on Data-flow analysis. The \emph{data-flow} 
    values design used in Control Flow for Phalanger 
    permits an effective representation with bit-vectors.
    In order to deal with the effects of routines 
    interaction, we used a modular approach, where 
    we analyse each routine only once in a generic 
    context setting to devise its return type and its 
    effects to the global state. This approach 
    can provide better scalability over 
    context sensitive analysis with memory 
    abstraction used by some of the other tools 
    for analysis of PHP code.
    
    The results of the type analysis can be used 
    by other tools, for example, a compiler or 
    integrated development environment plug-in. 
    The inferred type information, provided as 
    a result of the type analysis, supports 
    distinction between type information derived 
    (indirectly) from documentation comments and 
    from actual code, thus it supports both 
    use cases: the compiler can use the safe 
    type information derived only from code, 
    but IDE plug-in can take advantage of the 
    PHPDoc documentation comments.
    
    The Control Flow for Phalanger was evaluated on 
    three real world PHP projects. Although, the tool 
    does not perform context sensitive analysis and 
    does not use memory abstraction, it was still 
    capable of discovering several real issues with 
    a good ratio of false positives.
    
    This result may indicate that, were some imprecision 
    in the analysis results can be tolerated, the 
    modular approach we used can give results comparable 
    to those of tools that use more complex methods, 
    but with better scalability.

    \section{Future Work}
    
        \subsubsection*{Phalanger Integration}
        The next step in integration with the Phalanger project 
        is to run type analysis as part of the compilation 
        process and update the compiler back-end so that it uses the 
        available type information to emit more efficient code.
        Finally, the possible performance gain when running 
        PHP websites like WordPress should be evaluated.
        
        \subsubsection*{Arrays Support}
        Variables that can be of type array are analysed properly, 
        but the structure of the array is not analysed. Therefore 
        any time an element of an array is accessed, we do not 
        have any type information for it and have to assume 
        the worst -- it can be any type.
        
        Arrays in PHP are often used as ad-hoc structural 
        types like records in Pascal or structs in C. 
        In such case, the array is typically subscribed to 
        only by a set of known string constants, 
        which represents a good opportunity for 
        static analysis.
        
        One of the possible concepts for local arrays 
        analysis we would like to investigate further 
        and implement is based on the fact that arrays 
        in PHP have copy semantics as opposed to most 
        of the other programming languages. We can model each constant 
        index of an array as a separate local variable. 
        For example, for code \code{\$a['x']=3} we create 
        two local variables: \code{\$a} and \code{\$a@x} 
        and the type of \code{\$a} would be an array and 
        the type of \code{\$a@x} would be an integer. 
        Such representation would still permit us to easily use 
        bit-vectors as the \emph{data-flow} representation.
        
        \subsubsection*{Performance Evaluation and Tuning}
        Some of the design decisions in Control Flow for 
        Phalanger were made for performance reasons. The 
        design is done in a way that permits further 
        performance targeted improvements, but firstly 
        more advanced evaluation of the current performance 
        needs to be done.
        
        One of the possible enhancements we are 
        planning to implement is more efficient 
        type information representation. Type information 
        is represented using a 64 bit value, however, 
        we can go even further and represent the type 
        information with 8 bit number, which will be an index 
        into a table with all the possible type information 
        instances for one routine. This would give us 255 
        possible combinations of types, which we assume is 
        enough for most of the routines. Since every expression 
        node in AST is annotated with type information and 
        \emph{data-flow} values are arrays of 
        type information, we expect memory consumption 
        and possibly performance improvement.
        
        %\subsubsection*{Context-sensitive Analysis}
        