\chapter{Existing Software}

    It is not in the scope of this thesis to provide a comprehensive 
    overview of the work related to static analysis of PHP or type 
    inference in dynamic languages in general. Furthermore, those 
    topics are being actively researched, but Control Flow for 
    Phalanger's focus is to adapt and apply some of the results 
    of the research that are deemed to be useful for the 
    purposes of type analysis of PHP.

    \section{Security Vulnerabilities in Web Applications}
    
    Most of the existing work on static analysis of PHP is 
    focused on discovery of security vulnerabilities in 
    web applications that typically come from improperly 
    handled user input, also called taint-style vulnerabilities. 
    It is important for such analyses to be able to follow 
    the flow of data from global variables like \code{\$\_POST} 
    that contain user input, therefore more precise model of 
    heap memory is required so that flow of data in between 
    object instances can be analysed. Likewise, 
    context sensitivity is important in this case.
    
    An analysis for security vulnerabilities has also a different 
    model of usage. Such analysis can be run not so frequently, 
    for example, only before release or as a part of a continuous 
    build process. Interactive on-the-fly analysis in an 
    integrated development environment could also be a viable 
    use case, but typically not a main goal. Moreover, such analysis 
    is certainly not expected to be run every time the 
    application is to be compiled or interpreted.
    
    Some of the available tools for detecting taint-style 
    vulnerabilities in PHP are Pixy\cite{jovanovic2006pixy} and recently released 
    Weverca: Web Verification Tool\cite{hauzarhunting}.    
    
    \subsection{Weverca: Web Verification Tool}
    \note{Precise Memory Model with aliasing, CFG inlining, assume nodes, 
        supported PHP versions}

    \section{Type Inference}
    
    Type inference for dynamic languages is typically implemented 
    for the purposes of compiler or interpreter. A notable implementation 
    is type inference for PHP in Facebook's Hip Hop project \cite{zhao2012hiphop}, 
    which is a compiler from PHP to C++ and a custom intermediate language 
    that can be run in Hip Hop virtual machine. Hip Hop performs type 
    analysis in order to find single type for a variable, but does not attempt 
    to analyse the possible variable's types further as long as it 
    does not have only one single type. 
        
    There are implementations of type inference for other dynamic languages. 
    Ecstatic\cite{madsen2007ecstatic} is type inference for Ruby 
    implemented using control flow insensitive cartesian product algorithm. 
    Rubydust\cite{an2011dynamic} introduces a \emph{constraint based dynamic 
    type inference} that infers static types based on dynamic program 
    executions.

    \subsection{Phantm}
    
    Phantm\cite{kneuss2010phantm} is a tool for detection of type related 
    errors. From all the projects mentioned in this chapter, the aim of 
    Phantm is closest to our project, which is why we also used Phantm 
    for evaluation and compared its results with ours in 
    section \ref{phantmresults} Comparison to Phantm.
    
    Phantm uses semi-dynamic and semi-static analysis approach. The web 
    application in question is run up to a defined point, which is invocation 
    of special Phantm's function that collects data about the state of the application, 
    especially, values of global variables. This data is then used as an initial 
    state for static analysis. The dynamic part of the analysis is called bootstrapping. 
    This design illustrates that although type related errors can be searched for 
    in generic frameworks, libraries or, for example, command line utilities 
    written in PHP, Phantm's focus is on complete web applications.
    
    \note{Precise memory model, Routines inlining, ...}
    
    \note{NOTE: comparison on design level here. Discussion of what are the implications 
    of the design with respect to the results and comparison to our results -- 
    in section ``results''.}
    
    \note{QUESTION: is it enough if there will be approx. 1 page about Weverca, 
    and approx. 1 page about Phantm?}
    
    
    
    
    
    
