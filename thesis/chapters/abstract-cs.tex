S dynamickými programovacími jazyky je možné psát kód bez 
typové informace a typy proměnných se mohou měnit za běhu. 
Přestože se dynamické typování snadněji používá a je vhodné pro 
rychlé prototypování, dynamicky typovaný kód může být více 
náchylný k programátorským chybám a představuje nelehký úkol 
pro překladače nebo interpretry. Vývojáři často používají 
dokumentační komentáře pro explicitní uvedení typové informace, 
nicméně dostupné nástroje většinou nekontrolují shodu mezi 
komentáři a vlastním kódem. V této práci se zaměřujeme na 
jeden z nejpopulárnějších dynamických programovacích jazyků: PHP. 
V rámci této práce jsme vyvinuli framework pro statickou analýzu 
PHP kódu jako část projektu Phalanger -- překladače PHP do .NET. 
I když, tento framework podporuje jakýkoliv druh statické analýzy, 
implementovali jsme především typovou analýzu za účelem odhalení 
typových chyb a nekonzistence kódu s dokumentačními komentáři. 
S pomocí této analýzy jsme odhalili několik reálých chyb a 
nekonzistencí s dokumentací v kódu několika reálných PHP 
projektů s dobrým poměrem falešně pozitivních chyb.